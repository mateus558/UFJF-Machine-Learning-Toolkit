%% Generated by Sphinx.
\def\sphinxdocclass{report}
\documentclass[letterpaper,10pt,english]{sphinxmanual}
\ifdefined\pdfpxdimen
   \let\sphinxpxdimen\pdfpxdimen\else\newdimen\sphinxpxdimen
\fi \sphinxpxdimen=.75bp\relax
\ifdefined\pdfimageresolution
    \pdfimageresolution= \numexpr \dimexpr1in\relax/\sphinxpxdimen\relax
\fi
%% let collapsable pdf bookmarks panel have high depth per default
\PassOptionsToPackage{bookmarksdepth=5}{hyperref}

\PassOptionsToPackage{warn}{textcomp}
\usepackage[utf8]{inputenc}
\ifdefined\DeclareUnicodeCharacter
% support both utf8 and utf8x syntaxes
  \ifdefined\DeclareUnicodeCharacterAsOptional
    \def\sphinxDUC#1{\DeclareUnicodeCharacter{"#1}}
  \else
    \let\sphinxDUC\DeclareUnicodeCharacter
  \fi
  \sphinxDUC{00A0}{\nobreakspace}
  \sphinxDUC{2500}{\sphinxunichar{2500}}
  \sphinxDUC{2502}{\sphinxunichar{2502}}
  \sphinxDUC{2514}{\sphinxunichar{2514}}
  \sphinxDUC{251C}{\sphinxunichar{251C}}
  \sphinxDUC{2572}{\textbackslash}
\fi
\usepackage{cmap}
\usepackage[T1]{fontenc}
\usepackage{amsmath,amssymb,amstext}
\usepackage{babel}



\usepackage{tgtermes}
\usepackage{tgheros}
\renewcommand{\ttdefault}{txtt}



\usepackage[Bjarne]{fncychap}
\usepackage[,numfigreset=1,mathnumfig]{sphinx}

\fvset{fontsize=auto}
\usepackage{geometry}


% Include hyperref last.
\usepackage{hyperref}
% Fix anchor placement for figures with captions.
\usepackage{hypcap}% it must be loaded after hyperref.
% Set up styles of URL: it should be placed after hyperref.
\urlstyle{same}

\addto\captionsenglish{\renewcommand{\contentsname}{Table of contents}}

\usepackage{sphinxmessages}
\setcounter{tocdepth}{1}



\title{UFJF \sphinxhyphen{} Machine Learning Toolkit}
\date{Jul 02, 2021}
\release{0.51.1\sphinxhyphen{}beta.8}
\author{Mateus Coutinho Marim}
\newcommand{\sphinxlogo}{\vbox{}}
\renewcommand{\releasename}{Release}
\makeindex
\begin{document}

\pagestyle{empty}
\sphinxmaketitle
\pagestyle{plain}
\sphinxtableofcontents
\pagestyle{normal}
\phantomsection\label{\detokenize{index::doc}}


\sphinxAtStartPar
UFJF\sphinxhyphen{}MLTK is a cross\sphinxhyphen{}platform framework written in the C++ language for the development and
usage of machine learning algorithms, addresses several types of learning problems such as classification, regression,
clustering, feature selection, and ensemble learning. It aims to provide an always growing
set of algorithms and tools for machine learning researchers and enthusiasts in its projects.

\sphinxAtStartPar
\sphinxstylestrong{API Reference}

\sphinxAtStartPar
You can find the API Reference at our repository \sphinxhref{https://mateus558.github.io/UFJF-Machine-Learning-Toolkit/}{Gihub Pages}.

\sphinxAtStartPar
\sphinxstylestrong{Cite us}

\sphinxAtStartPar
If you use our project in your research, you can cite us by adding the bibtex from the \sphinxhref{https://www.researchgate.net/publication/333406079\_UFJF-MLTK\_a\_framework\_for\_machine\_learning\_algorithms}{project paper}:

\begin{sphinxVerbatim}[commandchars=\\\{\}]
@inproceedings\PYGZob{}10.1145/3330204.3330273,
   author = \PYGZob{}Marim, Mateus Coutinho and de Oliveira, Alessandreia Marta and Villela, Saulo Moraes\PYGZcb{},
   title = \PYGZob{}UFJF\PYGZhy{}MLTK: A Framework for Machine Learning Algorithms\PYGZcb{},
   year = \PYGZob{}2019\PYGZcb{},
   isbn = \PYGZob{}9781450372374\PYGZcb{},
   publisher = \PYGZob{}Association for Computing Machinery\PYGZcb{},
   address = \PYGZob{}New York, NY, USA\PYGZcb{},
   url = \PYGZob{}https://doi.org/10.1145/3330204.3330273\PYGZcb{},
   doi = \PYGZob{}10.1145/3330204.3330273\PYGZcb{},
   booktitle = \PYGZob{}Proceedings of the XV Brazilian Symposium on Information Systems\PYGZcb{},
   articleno = \PYGZob{}63\PYGZcb{},
   numpages = \PYGZob{}8\PYGZcb{},
   keywords = \PYGZob{}object\PYGZhy{}oriented programming, machine learning, Framework\PYGZcb{},
   location = \PYGZob{}Aracaju, Brazil\PYGZcb{},
   series = \PYGZob{}SBSI’19\PYGZcb{}
\PYGZcb{}
\end{sphinxVerbatim}

\sphinxAtStartPar
\sphinxstylestrong{Authors}
\begin{itemize}
\item {} 
\sphinxAtStartPar
Mateus Coutinho Marim

\item {} 
\sphinxAtStartPar
Saulo Moraes Villela

\item {} 
\sphinxAtStartPar
Alessandreia Oliveira

\end{itemize}

\sphinxAtStartPar
\sphinxstylestrong{Contact}

\sphinxAtStartPar
Feel free to contact me at any time to clear doubts that you could have and if you want to contribute to the development of the framework. You can contact me at my e\sphinxhyphen{}mail address \sphinxhref{mailto:mateus.marim@ice.ufjf.br}{mateus.marim@ice.ufjf.br}.


\chapter{Modules}
\label{\detokenize{overview/modules:modules}}\label{\detokenize{overview/modules::doc}}

\chapter{Architecture}
\label{\detokenize{overview/architecture:architecture}}\label{\detokenize{overview/architecture::doc}}

\chapter{Installation}
\label{\detokenize{getting_started/installation:installation}}\label{\detokenize{getting_started/installation::doc}}
\sphinxAtStartPar
As one of the main pourposes of UFJF\sphinxhyphen{}MLTK is the easy of use, the compilation and installation can’t be different, for
that the project was made using the cross\sphinxhyphen{}platform build management tool \sphinxcode{\sphinxupquote{cmake}} as most of the known C++ open source
projects.


\section{Requirements}
\label{\detokenize{getting_started/installation:requirements}}\begin{itemize}
\item {} 
\sphinxAtStartPar
CMake

\item {} 
\sphinxAtStartPar
C++ compiler with support to C++17

\item {} 
\sphinxAtStartPar
Gnuplot \textgreater{}= 5 (Optional, but needed for the Visualization module)

\end{itemize}


\section{Build on any system}
\label{\detokenize{getting_started/installation:build-on-any-system}}
\sphinxAtStartPar
The project can be compiled using the same commands in any system, the only difference is that on \sphinxcode{\sphinxupquote{Windows}} you’ll need
to make sure that the folder containing UFJF\sphinxhyphen{}MLTK is in your include path, so you can use include statements as
\sphinxcode{\sphinxupquote{\#include \textless{}ufjfmltk/Core.hpp\textgreater{}}}. For the standart instalation you only need to execute the following commands on
the project folder:

\begin{sphinxVerbatim}[commandchars=\\\{\}]
\PYG{n}{cmake} \PYG{o}{\PYGZhy{}}\PYG{n}{B} \PYG{n}{build}
\PYG{n}{cmake} \PYG{o}{\PYGZhy{}}\PYG{o}{\PYGZhy{}}\PYG{n}{build} \PYG{n}{build}
\end{sphinxVerbatim}

\sphinxAtStartPar
\sphinxcode{\sphinxupquote{CMake}} can generate projects for several IDEs, if you have more than one C++ IDE in your operational
system you can especify which one you want to use by adding the flag \sphinxhyphen{}G to CMake, for example, if you want to configure the
project for Visual Studio, you could execute the command as \sphinxcode{\sphinxupquote{cmake \sphinxhyphen{}B build \sphinxhyphen{}G "Visual Studio 16 2019"}} and then open the
generated project on it.

\sphinxAtStartPar
UFJF\sphinxhyphen{}MLTK was projected to be as modular as possible, so if you don’t want to compile some module, you could just turn
off it’s configuration on cmake, keeping in mind that it would be compiled in the same way if it’s a dependency for
another module to be compiled. The available options to be set on cmake are listed below:


\begin{savenotes}\sphinxattablestart
\centering
\begin{tabulary}{\linewidth}[t]{|T|T|T|}
\hline

\sphinxAtStartPar
CMake option
&
\sphinxAtStartPar
Default value
&
\sphinxAtStartPar
Description
\\
\hline
\sphinxAtStartPar
\sphinxhyphen{}DBUILD\_LIBVISUALIZE
&
\sphinxAtStartPar
ON
&
\sphinxAtStartPar
Tells if the visualization module must be built
\\
\hline
\sphinxAtStartPar
\sphinxhyphen{}DBUILD\_LIBCLASSIFIER
&
\sphinxAtStartPar
ON
&
\sphinxAtStartPar
Tells if the classifier module must be built
\\
\hline
\sphinxAtStartPar
\sphinxhyphen{}DBUILD\_LIBREGRESSOR
&
\sphinxAtStartPar
ON
&
\sphinxAtStartPar
Tells if the regressor module must be built
\\
\hline
\sphinxAtStartPar
\sphinxhyphen{}DBUILD\_LIBCLUSTERER
&
\sphinxAtStartPar
ON
&
\sphinxAtStartPar
Tells if the clusterer module must be built
\\
\hline
\sphinxAtStartPar
\sphinxhyphen{}DBUILD\_LIBFEATSELECT
&
\sphinxAtStartPar
ON
&
\sphinxAtStartPar
Tells if the feature selection module must be built
\\
\hline
\sphinxAtStartPar
\sphinxhyphen{}DBUILD\_LIBVALIDATION
&
\sphinxAtStartPar
ON
&
\sphinxAtStartPar
Tells if the validation module must be built
\\
\hline
\end{tabulary}
\par
\sphinxattableend\end{savenotes}


\section{Including to your CMake project}
\label{\detokenize{getting_started/installation:including-to-your-cmake-project}}
\sphinxAtStartPar
Following are minimal scripts to include ufjfmltk to your CMake project. The first method is by simply cloning
ufjfmltk repository into the main project folder and include it with \sphinxcode{\sphinxupquote{add\_subdirectory}}, it’s a good method if
you wish to use the latest updates on the framework, but it may break your application in future updates.

\begin{sphinxVerbatim}[commandchars=\\\{\}]
\PYG{n}{cmake\PYGZus{}minimum\PYGZus{}required}\PYG{p}{(}\PYG{n}{VERSION} \PYG{l+m+mf}{3.15}\PYG{p}{)}
\PYG{n}{project}\PYG{p}{(}\PYG{n}{project\PYGZus{}name}\PYG{p}{)}

\PYG{n}{set}\PYG{p}{(}\PYG{n}{CMAKE\PYGZus{}CXX\PYGZus{}STANDARD} \PYG{l+m+mi}{17}\PYG{p}{)}

\PYG{n}{add\PYGZus{}subdirectory}\PYG{p}{(}\PYG{n}{ufjfmltk}\PYG{p}{)}
\PYG{n}{add\PYGZus{}executable}\PYG{p}{(}\PYG{n}{project\PYGZus{}name} \PYG{n}{main}\PYG{p}{.}\PYG{n}{cpp}\PYG{p}{)}
\end{sphinxVerbatim}

\sphinxAtStartPar
The second and most recommended method is by using \sphinxcode{\sphinxupquote{FetchContent}}, with this approach you need to select one of the releases on the repository
and copy the link to its code \sphinxcode{\sphinxupquote{tar.gz}} file, this way you garantee that your project will work even when the framework receive major updates.

\begin{sphinxVerbatim}[commandchars=\\\{\}]
\PYG{n}{cmake\PYGZus{}minimum\PYGZus{}required}\PYG{p}{(}\PYG{n}{VERSION} \PYG{l+m+mf}{3.15}\PYG{p}{)}
\PYG{n}{project}\PYG{p}{(}\PYG{n}{project\PYGZus{}name}\PYG{p}{)}
\PYG{n}{set}\PYG{p}{(}\PYG{n}{CMAKE\PYGZus{}CXX\PYGZus{}STANDARD} \PYG{l+m+mi}{17}\PYG{p}{)}

\PYG{n}{include}\PYG{p}{(}\PYG{n}{FetchContent}\PYG{p}{)}
\PYG{n}{FetchContent\PYGZus{}Declare}\PYG{p}{(}
        \PYG{n}{ufjfmltk}
        \PYG{c+cp}{\PYGZsh{}}\PYG{c+cp}{ Specify the commit you depend on and update it regularly.}
        \PYG{n}{URL} \PYG{n+nl}{https}\PYG{p}{:}\PYG{c+c1}{//github.com/mateus558/UFJF\PYGZhy{}Machine\PYGZhy{}Learning\PYGZhy{}Toolkit/archive/refs/tags/v0.51.1\PYGZhy{}beta.7.tar.gz}
\PYG{p}{)}
\PYG{n}{FetchContent\PYGZus{}MakeAvailable}\PYG{p}{(}\PYG{n}{ufjfmltk}\PYG{p}{)}

\PYG{n}{add\PYGZus{}executable}\PYG{p}{(}\PYG{n}{project\PYGZus{}name} \PYG{n}{main}\PYG{p}{.}\PYG{n}{cpp}\PYG{p}{)}
\PYG{n}{target\PYGZus{}link\PYGZus{}libraries}\PYG{p}{(}\PYG{n}{project\PYGZus{}name} \PYG{n}{ufjfmltk}\PYG{p}{)}
\end{sphinxVerbatim}


\section{Adding UFJF\sphinxhyphen{}MLTK libraries to Windows environment}
\label{\detokenize{getting_started/installation:adding-ufjf-mltk-libraries-to-windows-environment}}
\sphinxAtStartPar
You need to enter into “System properties” and the environment variable CPATH with the value pointing to the folder containing the binaries, the default folder is “C:/UFJF\sphinxhyphen{}MLTK/bin”.


\section{Compiling your project including UFJF\sphinxhyphen{}MLTK}
\label{\detokenize{getting_started/installation:compiling-your-project-including-ufjf-mltk}}
\sphinxAtStartPar
With the libraries compiled and installed on the system you only need to add the UFJF\sphinxhyphen{}MLTK flag to the compiler to link the libraries to your program. Supose that we want to compile a source called \sphinxcode{\sphinxupquote{foo.cpp}} containing a main function, to compile it on the command line, you just need to add the flag \sphinxcode{\sphinxupquote{\sphinxhyphen{}lufjfmltk}}, for example, \sphinxcode{\sphinxupquote{g++ foo.cpp \sphinxhyphen{}o foo \sphinxhyphen{}lufjfmltk}} and on Windows \sphinxcode{\sphinxupquote{g++ foo.cpp \sphinxhyphen{}o foo \sphinxhyphen{}L\textless{}install\_folder\textgreater{} \sphinxhyphen{}lufjfmltk}}.

\sphinxAtStartPar
Unix systems: \sphinxcode{\sphinxupquote{g++ foo.cpp \sphinxhyphen{}o foo \sphinxhyphen{}lufjfmltk}}

\sphinxAtStartPar
Windows: \sphinxcode{\sphinxupquote{g++ foo.cpp \sphinxhyphen{}o foo \sphinxhyphen{}L\textless{}install\_folder\textgreater{} \sphinxhyphen{}lufjfmltk}}

\sphinxAtStartPar
With these steps complete, everything is set up and ready to use!


\section{Going through installers}
\label{\detokenize{getting_started/installation:going-through-installers}}
\sphinxAtStartPar
To make the framework installation easier for whom only whants to use the framework API, at each release are generated
installers that installs the framework and make it available to all system. You can find all \sphinxhref{https://github.com/mateus558/UFJF-Machine-Learning-Toolkit/releases}{releases here}.


\subsection{Ubuntu and Debian based OS}
\label{\detokenize{getting_started/installation:ubuntu-and-debian-based-os}}
\sphinxAtStartPar
Download the \sphinxcode{\sphinxupquote{.deb}} file corresponding to the desired framework release and execute the following command.

\begin{sphinxVerbatim}[commandchars=\\\{\}]
\PYG{n}{sudo} \PYG{n}{dpkg} \PYG{o}{\PYGZhy{}}\PYG{n}{i} \PYG{n}{ufjfmltk}\PYG{o}{\PYGZhy{}}\PYG{o}{\PYGZlt{}}\PYG{n}{version}\PYG{o}{\PYGZgt{}}\PYG{o}{\PYGZhy{}}\PYG{n}{Linux}\PYG{o}{\PYGZhy{}}\PYG{o}{\PYGZlt{}}\PYG{n}{cpu\PYGZus{}architecture}\PYG{o}{\PYGZgt{}}\PYG{p}{.}\PYG{n}{deb}
\end{sphinxVerbatim}


\subsection{Windows}
\label{\detokenize{getting_started/installation:windows}}
\begin{figure}[htbp]
\centering
\capstart

\noindent\sphinxincludegraphics[width=600\sphinxpxdimen]{{w1}.png}
\caption{1 \sphinxhyphen{} Click \sphinxstylestrong{Next} button.}\label{\detokenize{getting_started/installation:id1}}\end{figure}

\begin{figure}[htbp]
\centering
\capstart

\noindent\sphinxincludegraphics[width=450\sphinxpxdimen]{{w2}.png}
\caption{2 \sphinxhyphen{} Click \sphinxstylestrong{I agree} button.}\label{\detokenize{getting_started/installation:id2}}\end{figure}

\begin{figure}[htbp]
\centering
\capstart

\noindent\sphinxincludegraphics[width=450\sphinxpxdimen]{{w3}.png}
\caption{3 \sphinxhyphen{} Add ufjfmltk to system PATH so it’ll be available to all system and click \sphinxstylestrong{Next}.}\label{\detokenize{getting_started/installation:id3}}\end{figure}

\begin{figure}[htbp]
\centering
\capstart

\noindent\sphinxincludegraphics[width=450\sphinxpxdimen]{{w4}.png}
\caption{4 \sphinxhyphen{} Click \sphinxstylestrong{Next} button.}\label{\detokenize{getting_started/installation:id4}}\end{figure}

\begin{figure}[htbp]
\centering
\capstart

\noindent\sphinxincludegraphics[width=450\sphinxpxdimen]{{w5}.png}
\caption{5 \sphinxhyphen{} Check \sphinxstylestrong{Don’t create shortcuts} and click \sphinxstylestrong{Next} button.}\label{\detokenize{getting_started/installation:id5}}\end{figure}

\begin{figure}[htbp]
\centering
\capstart

\noindent\sphinxincludegraphics[width=450\sphinxpxdimen]{{w6}.png}
\caption{6 \sphinxhyphen{} Click \sphinxstylestrong{Install} button.}\label{\detokenize{getting_started/installation:id6}}\end{figure}

\begin{figure}[htbp]
\centering
\capstart

\noindent\sphinxincludegraphics[width=450\sphinxpxdimen]{{w7}.png}
\caption{7 \sphinxhyphen{} Click \sphinxstylestrong{Finish} button.}\label{\detokenize{getting_started/installation:id7}}\end{figure}


\subsection{Other linux based OS}
\label{\detokenize{getting_started/installation:other-linux-based-os}}
\sphinxAtStartPar
Download the \sphinxcode{\sphinxupquote{.run}} file corresponding to the desired framework release and follow these steps.

\begin{figure}[htbp]
\centering
\capstart

\noindent\sphinxincludegraphics[width=450\sphinxpxdimen]{{l1}.png}
\caption{1 \sphinxhyphen{} Click \sphinxstylestrong{Next} button.}\label{\detokenize{getting_started/installation:id8}}\end{figure}

\begin{figure}[htbp]
\centering
\capstart

\noindent\sphinxincludegraphics[width=450\sphinxpxdimen]{{l2}.png}
\caption{2\sphinxhyphen{} Choose where do you want to install ufjfmltk.}\label{\detokenize{getting_started/installation:id9}}\end{figure}

\begin{figure}[htbp]
\centering
\capstart

\noindent\sphinxincludegraphics[width=450\sphinxpxdimen]{{l3}.png}
\caption{3 \sphinxhyphen{} Click \sphinxstylestrong{Next} button.}\label{\detokenize{getting_started/installation:id10}}\end{figure}

\begin{figure}[htbp]
\centering
\capstart

\noindent\sphinxincludegraphics[width=450\sphinxpxdimen]{{l4}.png}
\caption{4 \sphinxhyphen{} Click \sphinxstylestrong{Install} button.}\label{\detokenize{getting_started/installation:id11}}\end{figure}

\begin{figure}[htbp]
\centering
\capstart

\noindent\sphinxincludegraphics[width=450\sphinxpxdimen]{{l5}.png}
\caption{5 \sphinxhyphen{} If you had choosen to install the framework in a system folder, you need to provide your \sphinxcode{\sphinxupquote{sudo}} password.}\label{\detokenize{getting_started/installation:id12}}\end{figure}

\begin{figure}[htbp]
\centering
\capstart

\noindent\sphinxincludegraphics[width=450\sphinxpxdimen]{{l6}.png}
\caption{6 \sphinxhyphen{} Click \sphinxstylestrong{Finish} button.}\label{\detokenize{getting_started/installation:id13}}\end{figure}


\chapter{Data management}
\label{\detokenize{getting_started/datamanagement:data-management}}\label{\detokenize{getting_started/datamanagement::doc}}
\sphinxAtStartPar
All the framework is composed of templates that make the development and use of ML (Machine Learning) algorithms easy and allow to the user to select the best available data type for the points of the data features.

\sphinxAtStartPar
One of the main concerns on ufjfmltk is to make ML algorithms development and usage simple, to accomplish that its necessary for the framework to have algorithms that can be applied on the underlying data structures,
one problem with that is that the user needs to get familiar with the available methods which can increase the framework learning curve. To solve that problem, data structures like the Data and Point templates are implemented
trying to keep compatibility with C++ STL library whethever is possible. With that, there’s no need to implement a, for example, sorting algorithm when there’s already the STL implementation for sorting operations.

\sphinxAtStartPar
In all the examples we’ll be using the double data type as default, but you can substitute by any type that makes sense for your problem.


\section{The Point template}
\label{\detokenize{getting_started/datamanagement:the-point-template}}
\sphinxAtStartPar
This class is a wrapper for the n\sphinxhyphen{}dimension variables of the dataset in the feature space, it also includes
along with the features values \sphinxstylestrong{X}, the target function \sphinxstylestrong{Y} and the \_alpha\_ weight associated to each point used in dual versions of ML algorithms.

\sphinxAtStartPar
In this section we’ll be learning the basics of a point manipulation and the operations that can be aplied to it, at the end I’ll also show
point expressions, a feature that can make the code easier to read and less error prone than the raw usage of loops.


\subsection{Initialization}
\label{\detokenize{getting_started/datamanagement:initialization}}\label{\detokenize{getting_started/datamanagement:pointusage}}
\sphinxAtStartPar
There are several ways to instantiate the Point template, the most simple instatiation is to just initialize the point
with \sphinxstyleemphasis{n} positions filled with zeros. The other properties can be initialized with its own methods.

\begin{sphinxVerbatim}[commandchars=\\\{\}]
\PYG{k+kt}{int} \PYG{n}{n} \PYG{o}{=} \PYG{l+m+mi}{3}\PYG{p}{,}
\PYG{n}{mltk}\PYG{o}{:}\PYG{o}{:}\PYG{n}{Point}\PYG{o}{\PYGZlt{}}\PYG{k+kt}{double}\PYG{o}{\PYGZgt{}} \PYG{n}{p}\PYG{p}{(}\PYG{n}{n}\PYG{p}{)}\PYG{p}{;}

\PYG{n}{p}\PYG{p}{.}\PYG{n}{Y}\PYG{p}{(}\PYG{p}{)} \PYG{o}{=} \PYG{l+m+mi}{1}\PYG{p}{;}
\PYG{n}{p}\PYG{p}{.}\PYG{n}{Alpha}\PYG{p}{(}\PYG{p}{)} \PYG{o}{=} \PYG{l+m+mi}{0}\PYG{p}{;}
\PYG{n}{p}\PYG{p}{.}\PYG{n}{Id}\PYG{p}{(}\PYG{p}{)} \PYG{o}{=} \PYG{l+m+mi}{5}\PYG{p}{;}
\end{sphinxVerbatim}

\sphinxAtStartPar
A more advanced initialization would be, for example, filling it with random values following a uniform distribution, something
that can be accomplished calling the \sphinxcode{\sphinxupquote{random\_init}} method.

\begin{sphinxVerbatim}[commandchars=\\\{\}]
\PYG{n}{mltk}\PYG{o}{:}\PYG{o}{:}\PYG{n}{Point}\PYG{o}{\PYGZlt{}}\PYG{o}{\PYGZgt{}} \PYG{n}{p1}\PYG{p}{;}

\PYG{n}{mltk}\PYG{o}{:}\PYG{o}{:}\PYG{n}{random\PYGZus{}init}\PYG{p}{(}\PYG{n}{p1}\PYG{p}{,} \PYG{l+m+mi}{3}\PYG{p}{,} \PYG{l+m+mi}{42}\PYG{p}{)}\PYG{p}{;}
\end{sphinxVerbatim}

\sphinxAtStartPar
In the code snipet above we initialized \sphinxstyleemphasis{p1} with 3 dimensions and values following a uniform random distribution with seed 42.
Bellow we can see another example using a different distribution.

\begin{sphinxVerbatim}[commandchars=\\\{\}]
\PYG{n}{mltk}\PYG{o}{:}\PYG{o}{:}\PYG{n}{Point}\PYG{o}{\PYGZlt{}}\PYG{k+kt}{double}\PYG{o}{\PYGZgt{}} \PYG{n}{p1}\PYG{p}{;}

\PYG{n}{p1} \PYG{o}{=} \PYG{n}{mltk}\PYG{o}{:}\PYG{o}{:}\PYG{n}{random\PYGZus{}init}\PYG{o}{\PYGZlt{}}\PYG{k+kt}{double}\PYG{p}{,} \PYG{n}{std}\PYG{o}{:}\PYG{o}{:}\PYG{n}{fisher\PYGZus{}f\PYGZus{}distribution}\PYG{o}{\PYGZlt{}}\PYG{k+kt}{double}\PYG{o}{\PYGZgt{}}\PYG{o}{\PYGZgt{}}\PYG{p}{(}\PYG{l+m+mf}{2.0}\PYG{p}{,} \PYG{l+m+mf}{2.0}\PYG{p}{,} \PYG{l+m+mi}{3}\PYG{p}{,} \PYG{l+m+mi}{42}\PYG{p}{)}\PYG{p}{;}
\end{sphinxVerbatim}


\subsection{Point expressions and operations}
\label{\detokenize{getting_started/datamanagement:point-expressions-and-operations}}
\sphinxAtStartPar
The main advantages of the Point template is the possibility to write code similar to math equations, taking away the burden of writting complex loops
to implement functions. Point expressions are made possible through the implementation of a expression template, it makes possible to overload the math equations
with similar sintax as the paper written version without losing performance when compiler optimizations are turned on (\sphinxcode{\sphinxupquote{\sphinxhyphen{}O3}}). Even on debug mode, the performance loss is acceptable
given that the code is less error prone and easier to read. One example of application is on the implementation of distance metrics like the euclidean distance.

\sphinxAtStartPar
Euclidean distance:
\begin{equation*}
\begin{split}d = \sqrt { \sum_{i=0}^{n-1} (p1_{i} - p2_{i})^{2}}\end{split}
\end{equation*}
\begin{sphinxVerbatim}[commandchars=\\\{\}]
\PYG{n}{mltk}\PYG{o}{:}\PYG{o}{:}\PYG{n}{Point}\PYG{o}{\PYGZlt{}}\PYG{o}{\PYGZgt{}} \PYG{n}{p1}\PYG{p}{(}\PYG{l+m+mi}{3}\PYG{p}{,} \PYG{l+m+mi}{2}\PYG{p}{)}\PYG{p}{,} \PYG{n}{p2}\PYG{p}{(}\PYG{l+m+mi}{3}\PYG{p}{,} \PYG{l+m+mf}{1.5}\PYG{p}{)}\PYG{p}{;}

\PYG{k+kt}{double} \PYG{n}{d} \PYG{o}{=} \PYG{n}{std}\PYG{o}{:}\PYG{o}{:}\PYG{n}{sqrt}\PYG{p}{(}\PYG{n}{mltk}\PYG{o}{:}\PYG{o}{:}\PYG{n}{pow}\PYG{p}{(}\PYG{n}{p1} \PYG{o}{\PYGZhy{}} \PYG{n}{p2}\PYG{p}{,} \PYG{l+m+mi}{2}\PYG{p}{)}\PYG{p}{.}\PYG{n}{sum}\PYG{p}{(}\PYG{p}{)}\PYG{p}{)}\PYG{p}{;} \PYG{c+c1}{// Euclidean distance between p1 and p2}
\end{sphinxVerbatim}

\sphinxAtStartPar
Another advantage is that the equation is lazy evaluated, i.e only evaluated when it’s result is required, like when doing a point assignment. This becomes an advantage on multithreaded environments, when
we want to evaluate an equation when a thread requires it.

\sphinxAtStartPar
Alongside with the Point template there are some operations that can be applied on the points, they are the basic arithmetic operations and some common math functions.

\sphinxAtStartPar
Examples of arithmetic operations:

\begin{sphinxVerbatim}[commandchars=\\\{\}]
\PYG{n}{mltk}\PYG{o}{:}\PYG{o}{:}\PYG{n}{Point}\PYG{o}{\PYGZlt{}}\PYG{o}{\PYGZgt{}} \PYG{n}{a}\PYG{p}{(}\PYG{l+m+mi}{3}\PYG{p}{,} \PYG{l+m+mf}{1.0}\PYG{p}{)}\PYG{p}{,} \PYG{n}{b}\PYG{p}{(}\PYG{l+m+mi}{3}\PYG{p}{,} \PYG{l+m+mf}{2.0}\PYG{p}{)}\PYG{p}{;}
\PYG{n}{mltk}\PYG{o}{:}\PYG{o}{:}\PYG{n}{Point}\PYG{o}{\PYGZlt{}}\PYG{o}{\PYGZgt{}} \PYG{n}{c} \PYG{o}{=} \PYG{n}{a} \PYG{o}{+} \PYG{n}{b}\PYG{p}{;}

\PYG{n}{b} \PYG{o}{=} \PYG{n}{a} \PYG{o}{\PYGZhy{}} \PYG{n}{c}\PYG{p}{;}
\PYG{n}{a} \PYG{o}{=} \PYG{n}{c} \PYG{o}{/} \PYG{l+m+mi}{2}\PYG{p}{;}
\PYG{n}{c} \PYG{o}{*}\PYG{o}{=} \PYG{l+m+mi}{3}\PYG{p}{;}
\end{sphinxVerbatim}

\sphinxAtStartPar
Also were implemented on the framework common math functions that can be aplied directly to the point objects.

\begin{sphinxVerbatim}[commandchars=\\\{\}]
\PYG{n}{mltk}\PYG{o}{:}\PYG{o}{:}\PYG{n}{Point}\PYG{o}{\PYGZlt{}}\PYG{o}{\PYGZgt{}} \PYG{n}{a}\PYG{p}{(}\PYG{l+m+mi}{3}\PYG{p}{,} \PYG{l+m+mf}{1.0}\PYG{p}{)}\PYG{p}{,} \PYG{n}{b}\PYG{p}{(}\PYG{l+m+mi}{3}\PYG{p}{,} \PYG{l+m+mf}{2.0}\PYG{p}{)}\PYG{p}{;}
\PYG{n}{mltk}\PYG{o}{:}\PYG{o}{:}\PYG{n}{Point}\PYG{o}{\PYGZlt{}}\PYG{o}{\PYGZgt{}} \PYG{n}{c} \PYG{o}{=} \PYG{n}{mltk}\PYG{o}{:}\PYG{o}{:}\PYG{n}{sin}\PYG{p}{(}\PYG{n}{a}\PYG{p}{)} \PYG{o}{+} \PYG{l+m+mi}{2} \PYG{o}{*} \PYG{n}{mltk}\PYG{o}{:}\PYG{o}{:}\PYG{n}{cos}\PYG{p}{(}\PYG{n}{b}\PYG{p}{)}\PYG{p}{;}

\PYG{k}{auto} \PYG{n}{d} \PYG{o}{=} \PYG{n}{mltk}\PYG{o}{:}\PYG{o}{:}\PYG{n}{pow}\PYG{p}{(}\PYG{n}{mltk}\PYG{o}{:}\PYG{o}{:}\PYG{n}{exp}\PYG{p}{(}\PYG{n}{c}\PYG{p}{)}\PYG{p}{,} \PYG{l+m+mi}{3}\PYG{p}{)}\PYG{p}{;}
\PYG{k+kt}{double} \PYG{n}{sum} \PYG{o}{=} \PYG{n}{d}\PYG{p}{.}\PYG{n}{sum}\PYG{p}{(}\PYG{p}{)}\PYG{p}{;}
\end{sphinxVerbatim}

\sphinxAtStartPar
This example above only shows a subset of operations that can be applied, for more you can see the list below.
\begin{itemize}
\item {} 
\sphinxAtStartPar
abs \sphinxcode{\sphinxupquote{mltk::abs(p)}} \sphinxhyphen{} absolute values;

\item {} 
\sphinxAtStartPar
max \sphinxcode{\sphinxupquote{mltk::max(p)}} \sphinxhyphen{} maximum value;

\item {} 
\sphinxAtStartPar
min \sphinxcode{\sphinxupquote{mltk::min(p)}} \sphinxhyphen{} minimum value;

\item {} 
\sphinxAtStartPar
norm \sphinxcode{\sphinxupquote{mltk::norm(p, norm\_type)}} \sphinxhyphen{} computes the norm of a point, by default \sphinxcode{\sphinxupquote{norm\_value = 2}};

\item {} 
\sphinxAtStartPar
dot \sphinxcode{\sphinxupquote{mltk::dot(p, q)}} \sphinxhyphen{} computes the dot product between \sphinxstyleemphasis{p} and \sphinxstyleemphasis{q};

\item {} 
\sphinxAtStartPar
log \sphinxcode{\sphinxupquote{mltk::log(p)}} \sphinxhyphen{} natural log values;

\item {} 
\sphinxAtStartPar
normalize \sphinxcode{\sphinxupquote{mltk::normalize(p, norm\_type)}} \sphinxhyphen{} normalize a point, by default \sphinxcode{\sphinxupquote{norm\_value = 2}};

\item {} 
\sphinxAtStartPar
linspace \sphinxcode{\sphinxupquote{mltk::linspace(lower, upper, N)}} \sphinxhyphen{} returns a point with \sphinxstyleemphasis{N} linear values from \sphinxstyleemphasis{lower} to \sphinxstyleemphasis{upper}.

\end{itemize}

\sphinxAtStartPar
We can also print the point content using the stream overload operator.

\begin{sphinxVerbatim}[commandchars=\\\{\}]
\PYG{n}{std}\PYG{o}{:}\PYG{o}{:}\PYG{n}{cout} \PYG{o}{\PYGZlt{}}\PYG{o}{\PYGZlt{}} \PYG{n}{p} \PYG{o}{\PYGZlt{}}\PYG{o}{\PYGZlt{}} \PYG{n}{std}\PYG{o}{:}\PYG{o}{:}\PYG{n}{endl}\PYG{p}{;}
\end{sphinxVerbatim}

\sphinxAtStartPar
You can access the features values of a point accessing the elemens of the \sphinxstylestrong{x} vector member or by treating the point as a container:

\begin{sphinxVerbatim}[commandchars=\\\{\}]
\PYG{k+kt}{int} \PYG{n}{i}\PYG{p}{,} \PYG{n}{dim} \PYG{o}{=} \PYG{n}{p}\PYG{p}{.}\PYG{n}{x}\PYG{p}{.}\PYG{n}{size}\PYG{p}{(}\PYG{p}{)}\PYG{p}{;}

\PYG{k}{for}\PYG{p}{(}\PYG{n}{i} \PYG{o}{=} \PYG{l+m+mi}{0}\PYG{p}{;} \PYG{n}{i} \PYG{o}{\PYGZlt{}} \PYG{n}{dim}\PYG{p}{;} \PYG{n}{i}\PYG{o}{+}\PYG{o}{+}\PYG{p}{)}\PYG{p}{\PYGZob{}}
    \PYG{n}{std}\PYG{o}{:}\PYG{o}{:}\PYG{n}{cout} \PYG{o}{\PYGZlt{}}\PYG{o}{\PYGZlt{}} \PYG{n}{p}\PYG{p}{[}\PYG{n}{i}\PYG{p}{]} \PYG{o}{\PYGZlt{}}\PYG{o}{\PYGZlt{}} \PYG{n}{std}\PYG{o}{:}\PYG{o}{:}\PYG{n}{endl}\PYG{p}{;}
\PYG{p}{\PYGZcb{}}

\PYG{c+c1}{// using iterators}
\PYG{k}{for}\PYG{p}{(}\PYG{k}{auto} \PYG{n}{it} \PYG{o}{=} \PYG{n}{p}\PYG{p}{.}\PYG{n}{begin}\PYG{p}{(}\PYG{p}{)}\PYG{p}{;} \PYG{n}{it} \PYG{o}{!}\PYG{o}{=} \PYG{n}{p}\PYG{p}{.}\PYG{n}{end}\PYG{p}{(}\PYG{p}{)}\PYG{p}{;} \PYG{n}{it}\PYG{o}{+}\PYG{o}{+}\PYG{p}{)}\PYG{p}{\PYGZob{}}
    \PYG{n}{std}\PYG{o}{:}\PYG{o}{:}\PYG{n}{cout} \PYG{o}{\PYGZlt{}}\PYG{o}{\PYGZlt{}} \PYG{p}{(}\PYG{o}{*}\PYG{n}{it}\PYG{p}{)} \PYG{o}{\PYGZlt{}}\PYG{o}{\PYGZlt{}} \PYG{n}{std}\PYG{o}{:}\PYG{o}{:}\PYG{n}{endl}\PYG{p}{;}
\PYG{p}{\PYGZcb{}}
\end{sphinxVerbatim}


\subsection{Algorithms}
\label{\detokenize{getting_started/datamanagement:algorithms}}
\sphinxAtStartPar
As we are keeping the compatibility with STL, there are several algorithms that are supported by the framework, for example if we want to
fill a Point with integers we can use the \sphinxcode{\sphinxupquote{std::iota}} algorithm for that, like standart C++ containers.

\begin{sphinxVerbatim}[commandchars=\\\{\}]
\PYG{n}{mltk}\PYG{o}{:}\PYG{o}{:}\PYG{n}{Point}\PYG{o}{\PYGZlt{}}\PYG{o}{\PYGZgt{}} \PYG{n}{p}\PYG{p}{(}\PYG{l+m+mi}{5}\PYG{p}{)}\PYG{p}{;}

\PYG{n}{std}\PYG{o}{:}\PYG{o}{:}\PYG{n}{iota}\PYG{p}{(}\PYG{n}{p}\PYG{p}{.}\PYG{n}{begin}\PYG{p}{(}\PYG{p}{)}\PYG{p}{,} \PYG{n}{p}\PYG{p}{.}\PYG{n}{end}\PYG{p}{(}\PYG{p}{)}\PYG{p}{,} \PYG{l+m+mi}{1}\PYG{p}{)}\PYG{p}{;}
\end{sphinxVerbatim}

\sphinxAtStartPar
This code will fill \sphinxstyleemphasis{p} with values ranging from 1 to 5.

\sphinxAtStartPar
We also could initialize a point with random values ranging from 1 to 10 and sort it after.

\begin{sphinxVerbatim}[commandchars=\\\{\}]
\PYG{n}{mltk}\PYG{o}{:}\PYG{o}{:}\PYG{n}{Point}\PYG{o}{\PYGZlt{}}\PYG{k+kt}{int}\PYG{o}{\PYGZgt{}} \PYG{n}{p}\PYG{p}{(}\PYG{l+m+mi}{5}\PYG{p}{)}\PYG{p}{;}

\PYG{n}{p} \PYG{o}{=} \PYG{n}{mltk}\PYG{o}{:}\PYG{o}{:}\PYG{n}{random\PYGZus{}init}\PYG{o}{\PYGZlt{}}\PYG{k+kt}{int}\PYG{p}{,} \PYG{n}{std}\PYG{o}{:}\PYG{o}{:}\PYG{n}{uniform\PYGZus{}int\PYGZus{}distribution}\PYG{o}{\PYGZlt{}}\PYG{k+kt}{int}\PYG{o}{\PYGZgt{}}\PYG{o}{\PYGZgt{}}\PYG{p}{(}\PYG{l+m+mi}{1}\PYG{p}{,} \PYG{l+m+mi}{10}\PYG{p}{,} \PYG{l+m+mi}{5}\PYG{p}{,} \PYG{l+m+mi}{42}\PYG{p}{)}\PYG{p}{;}
\PYG{n}{std}\PYG{o}{:}\PYG{o}{:}\PYG{n}{sort}\PYG{p}{(}\PYG{n}{p}\PYG{p}{.}\PYG{n}{begin}\PYG{p}{(}\PYG{p}{)}\PYG{p}{,} \PYG{n}{p}\PYG{p}{.}\PYG{n}{end}\PYG{p}{(}\PYG{p}{)}\PYG{p}{)}\PYG{p}{;}
\end{sphinxVerbatim}


\section{The Data template}
\label{\detokenize{getting_started/datamanagement:the-data-template}}
\sphinxAtStartPar
As we’re normally dealing with datasets we have multiple points to work, so there’s the necessity to have a class to wrap all the information about this dataset and the operations that we can apply to these data. As the
Point the Data template is also compatible with STL algorithms.

\sphinxAtStartPar
These are the supported formats to load datasets:
\begin{itemize}
\item {} 
\sphinxAtStartPar
arff

\item {} 
\sphinxAtStartPar
csv

\item {} 
\sphinxAtStartPar
data

\item {} 
\sphinxAtStartPar
txt (Embrapa datasets format)

\end{itemize}


\subsection{Memory sharing between Data objects}
\label{\detokenize{getting_started/datamanagement:memory-sharing-between-data-objects}}\label{\detokenize{getting_started/datamanagement:datamemorysharing}}
\sphinxAtStartPar
Sometimes we need to run several algorithms in the same dataset and, if we’ll not transform the feature space of the variables, copying all the data to each algorithm that we’ll run can be a waste of memory and at sometimes a simple computer can’t handle the memory consumption.
Thinking in that the Data class was developed using smart pointers, a tool introduced at C++11 that handles the sharing of memory between objects with almost the same speed of raw pointers, but memory safe.

\sphinxAtStartPar
Because of that an array of points in the data class is defined with T as a generic data type as:

\begin{sphinxVerbatim}[commandchars=\\\{\}]
\PYG{n}{std}\PYG{o}{:}\PYG{o}{:}\PYG{n}{vector}\PYG{o}{\PYGZlt{}}\PYG{n}{std}\PYG{o}{:}\PYG{o}{:}\PYG{n}{shared\PYGZus{}ptr}\PYG{o}{\PYGZlt{}}\PYG{n}{Point}\PYG{o}{\PYGZlt{}} \PYG{n}{T} \PYG{o}{\PYGZgt{}} \PYG{o}{\PYGZgt{}} \PYG{o}{\PYGZgt{}} \PYG{n}{points}\PYG{p}{;}
\end{sphinxVerbatim}

\sphinxAtStartPar
So if you use the = operator with other data object, they will be point to the same memory space of the original object, to make a deep copy the content of an object to another you’ll have to use the \sphinxcode{\sphinxupquote{copy()}} method.

\begin{sphinxVerbatim}[commandchars=\\\{\}]
\PYG{n}{Data}\PYG{o}{\PYGZlt{}}\PYG{k+kt}{double}\PYG{o}{\PYGZgt{}} \PYG{n}{other}\PYG{p}{;}

\PYG{n}{other} \PYG{o}{=} \PYG{n}{data}\PYG{p}{.}\PYG{n}{copy}\PYG{p}{(}\PYG{p}{)}
\end{sphinxVerbatim}


\subsection{Loading a dataset to a Data object}
\label{\detokenize{getting_started/datamanagement:loading-a-dataset-to-a-data-object}}\label{\detokenize{getting_started/datamanagement:loadingdataset}}
\sphinxAtStartPar
This can be easily done with the Data class initialization, accomplished with only one line of code.

\begin{sphinxVerbatim}[commandchars=\\\{\}]
\PYG{n}{Data}\PYG{o}{\PYGZlt{}}\PYG{k+kt}{double}\PYG{o}{\PYGZgt{}} \PYG{n}{data}\PYG{p}{(}\PYG{l+s}{\PYGZdq{}}\PYG{l+s}{wine.csv}\PYG{l+s}{\PYGZdq{}}\PYG{p}{)}\PYG{p}{;}
\end{sphinxVerbatim}

\sphinxAtStartPar
Or if you want the data object initially empty.

\begin{sphinxVerbatim}[commandchars=\\\{\}]
\PYG{n}{Data}\PYG{o}{\PYGZlt{}}\PYG{k+kt}{double}\PYG{o}{\PYGZgt{}} \PYG{n}{data}\PYG{p}{;}

\PYG{n}{data}\PYG{p}{.}\PYG{n}{load}\PYG{p}{(}\PYG{l+s}{\PYGZdq{}}\PYG{l+s}{wine.csv}\PYG{l+s}{\PYGZdq{}}\PYG{p}{)}\PYG{p}{;}
\end{sphinxVerbatim}

\sphinxAtStartPar
If the target function value or expected value is at the end of the dataset, it must be informed to the constructor.

\begin{sphinxVerbatim}[commandchars=\\\{\}]
\PYG{n}{Data}\PYG{o}{\PYGZlt{}}\PYG{k+kt}{double}\PYG{o}{\PYGZgt{}} \PYG{n}{data}\PYG{p}{(}\PYG{l+s}{\PYGZdq{}}\PYG{l+s}{wine.arff}\PYG{l+s}{\PYGZdq{}}\PYG{p}{,} \PYG{n+nb}{true}\PYG{p}{)}\PYG{p}{;}
\end{sphinxVerbatim}

\sphinxAtStartPar
Note that in all formats the target function must be at the beginning or at the end of each line of the file. You can print
all the dataset with the C++ standard output stream operator.

\begin{sphinxVerbatim}[commandchars=\\\{\}]
\PYG{n}{Data}\PYG{o}{\PYGZlt{}}\PYG{k+kt}{double}\PYG{o}{\PYGZgt{}} \PYG{n}{data}\PYG{p}{(}\PYG{l+s}{\PYGZdq{}}\PYG{l+s}{wine.csv}\PYG{l+s}{\PYGZdq{}}\PYG{p}{)}\PYG{p}{;}

\PYG{n}{std}\PYG{o}{:}\PYG{o}{:}\PYG{n}{cout} \PYG{o}{\PYGZlt{}}\PYG{o}{\PYGZlt{}} \PYG{n}{data} \PYG{o}{\PYGZlt{}}\PYG{o}{\PYGZlt{}} \PYG{n}{std}\PYG{o}{:}\PYG{o}{:}\PYG{n}{endl}\PYG{p}{;}
\end{sphinxVerbatim}


\subsection{Getting information about the dataset}
\label{\detokenize{getting_started/datamanagement:getting-information-about-the-dataset}}\label{\detokenize{getting_started/datamanagement:datasetinformation}}
\sphinxAtStartPar
After the data is loaded into the memory, we can get some useful information about the data.

\begin{sphinxVerbatim}[commandchars=\\\{\}]
\PYG{n}{std}\PYG{o}{:}\PYG{o}{:}\PYG{n}{cout} \PYG{o}{\PYGZlt{}}\PYG{o}{\PYGZlt{}} \PYG{l+s}{\PYGZdq{}}\PYG{l+s}{Dataset information: }\PYG{l+s}{\PYGZdq{}} \PYG{o}{\PYGZlt{}}\PYG{o}{\PYGZlt{}} \PYG{n}{std}\PYG{o}{:}\PYG{o}{:}\PYG{n}{endl}\PYG{p}{;}
\PYG{n}{std}\PYG{o}{:}\PYG{o}{:}\PYG{n}{cout} \PYG{o}{\PYGZlt{}}\PYG{o}{\PYGZlt{}} \PYG{l+s}{\PYGZdq{}}\PYG{l+s}{Number of points: }\PYG{l+s}{\PYGZdq{}} \PYG{o}{\PYGZlt{}}\PYG{o}{\PYGZlt{}} \PYG{n}{data}\PYG{p}{.}\PYG{n}{size}\PYG{p}{(}\PYG{p}{)} \PYG{o}{\PYGZlt{}}\PYG{o}{\PYGZlt{}} \PYG{n}{std}\PYG{o}{:}\PYG{o}{:}\PYG{n}{endl}\PYG{p}{;}
\PYG{n}{std}\PYG{o}{:}\PYG{o}{:}\PYG{n}{cout} \PYG{o}{\PYGZlt{}}\PYG{o}{\PYGZlt{}} \PYG{l+s}{\PYGZdq{}}\PYG{l+s}{Dimension: }\PYG{l+s}{\PYGZdq{}} \PYG{o}{\PYGZlt{}}\PYG{o}{\PYGZlt{}} \PYG{n}{data}\PYG{p}{.}\PYG{n}{dim}\PYG{p}{(}\PYG{p}{)} \PYG{o}{\PYGZlt{}}\PYG{o}{\PYGZlt{}} \PYG{n}{std}\PYG{o}{:}\PYG{o}{:}\PYG{n}{endl}\PYG{p}{;}
\PYG{n}{std}\PYG{o}{:}\PYG{o}{:}\PYG{n}{cout} \PYG{o}{\PYGZlt{}}\PYG{o}{\PYGZlt{}} \PYG{l+s}{\PYGZdq{}}\PYG{l+s}{Classes: }\PYG{l+s}{\PYGZdq{}} \PYG{o}{\PYGZlt{}}\PYG{o}{\PYGZlt{}} \PYG{n}{data}\PYG{p}{.}\PYG{n}{classes}\PYG{p}{(}\PYG{p}{)} \PYG{o}{\PYGZlt{}}\PYG{o}{\PYGZlt{}} \PYG{n}{std}\PYG{o}{:}\PYG{o}{:}\PYG{n}{endl}\PYG{p}{;}
\PYG{n}{std}\PYG{o}{:}\PYG{o}{:}\PYG{n}{cout} \PYG{o}{\PYGZlt{}}\PYG{o}{\PYGZlt{}} \PYG{l+s}{\PYGZdq{}}\PYG{l+s}{Classes distribution: }\PYG{l+s}{\PYGZdq{}} \PYG{o}{\PYGZlt{}}\PYG{o}{\PYGZlt{}} \PYG{n}{data}\PYG{p}{.}\PYG{n}{classesDistribution}\PYG{p}{(}\PYG{p}{)} \PYG{o}{\PYGZlt{}}\PYG{o}{\PYGZlt{}} \PYG{n}{std}\PYG{o}{:}\PYG{o}{:}\PYG{n}{endl}\PYG{p}{;}
\end{sphinxVerbatim}


\subsection{Scanning through the data points}
\label{\detokenize{getting_started/datamanagement:scanning-through-the-data-points}}\label{\detokenize{getting_started/datamanagement:scanningpoints}}
\sphinxAtStartPar
There are two ways to access the points contained on a Data object, the first is the operator \sphinxcode{\sphinxupquote{{[}{]}}} that returns a smart pointer to a point contained in the Data object, the other
way is through the function call operator \sphinxcode{\sphinxupquote{()}} that returns a reference to the Point object. Almost all the times we would want to use the second option to avoid the pointer sintax.

\sphinxAtStartPar
Though the smart pointers are intended to be preferred in the place of the raw pointers, they work almost the same way as we are used with the classic pointers, so there’s no much difference in this.

\sphinxAtStartPar
In this example we’ll see how we can print each point of the dataset:

\begin{sphinxVerbatim}[commandchars=\\\{\}]
\PYG{k+kt}{int} \PYG{n}{i}\PYG{p}{,} \PYG{n}{j}\PYG{p}{,} \PYG{n}{size} \PYG{o}{=} \PYG{n}{data}\PYG{p}{.}\PYG{n}{size}\PYG{p}{(}\PYG{p}{)}\PYG{p}{,} \PYG{n}{dim} \PYG{o}{=} \PYG{n}{data}\PYG{p}{.}\PYG{n}{dim}\PYG{p}{(}\PYG{p}{)}\PYG{p}{;}

\PYG{k}{for}\PYG{p}{(}\PYG{n}{i} \PYG{o}{=} \PYG{l+m+mi}{0}\PYG{p}{;} \PYG{n}{i} \PYG{o}{\PYGZlt{}} \PYG{n}{size}\PYG{p}{;} \PYG{n}{i}\PYG{o}{+}\PYG{o}{+}\PYG{p}{)}\PYG{p}{\PYGZob{}}
    \PYG{n}{std}\PYG{o}{:}\PYG{o}{:}\PYG{n}{cout} \PYG{o}{\PYGZlt{}}\PYG{o}{\PYGZlt{}} \PYG{n}{data}\PYG{p}{(}\PYG{n}{i}\PYG{p}{)} \PYG{o}{\PYGZlt{}}\PYG{o}{\PYGZlt{}} \PYG{n}{std}\PYG{o}{:}\PYG{o}{:}\PYG{n}{endl}\PYG{p}{;}
\PYG{p}{\PYGZcb{}}
\end{sphinxVerbatim}

\sphinxAtStartPar
Treating the Data object as a container:

\begin{sphinxVerbatim}[commandchars=\\\{\}]
\PYG{k}{for}\PYG{p}{(}\PYG{n}{i} \PYG{o}{=} \PYG{l+m+mi}{0}\PYG{p}{;} \PYG{n}{i} \PYG{o}{\PYGZlt{}} \PYG{n}{size}\PYG{p}{;} \PYG{n}{i}\PYG{o}{+}\PYG{o}{+}\PYG{p}{)}\PYG{p}{\PYGZob{}}
    \PYG{k}{for}\PYG{p}{(}\PYG{n}{j} \PYG{o}{=} \PYG{l+m+mi}{0}\PYG{p}{;} \PYG{n}{j} \PYG{o}{\PYGZlt{}} \PYG{n}{dim}\PYG{p}{;} \PYG{n}{j}\PYG{o}{+}\PYG{o}{+}\PYG{p}{)}
        \PYG{n}{std}\PYG{o}{:}\PYG{o}{:}\PYG{n}{cout} \PYG{o}{\PYGZlt{}}\PYG{o}{\PYGZlt{}} \PYG{n}{data}\PYG{p}{(}\PYG{n}{i}\PYG{p}{)}\PYG{p}{[}\PYG{n}{j}\PYG{p}{]} \PYG{o}{\PYGZlt{}}\PYG{o}{\PYGZlt{}} \PYG{n}{std}\PYG{o}{:}\PYG{o}{:}\PYG{n}{endl}\PYG{p}{;}
\PYG{p}{\PYGZcb{}}
\end{sphinxVerbatim}


\subsection{Applying transformations to data}
\label{\detokenize{getting_started/datamanagement:applying-transformations-to-data}}
\sphinxAtStartPar
Often we dont want only to load the data but also want to apply transformations to it, be it a point/feature removal or a preprocessing step.
For it the Data template provide methods for point and features removal/insertion and the method \sphinxcode{\sphinxupquote{apply}} that allows to apply a function to
the points contained in the object.

\sphinxAtStartPar
We could normalize the dataset points like this, instead of looping through the points:

\begin{sphinxVerbatim}[commandchars=\\\{\}]
\PYG{n}{mltk}\PYG{o}{:}\PYG{o}{:}\PYG{n}{Data}\PYG{o}{\PYGZlt{}}\PYG{o}{\PYGZgt{}} \PYG{n}{data}\PYG{p}{(}\PYG{l+s}{\PYGZdq{}}\PYG{l+s}{iris.csv}\PYG{l+s}{\PYGZdq{}}\PYG{p}{)}\PYG{p}{;}

\PYG{k}{auto} \PYG{n}{normalization} \PYG{o}{=} \PYG{p}{[}\PYG{p}{]}\PYG{p}{(}\PYG{n}{mltk}\PYG{o}{:}\PYG{o}{:}\PYG{n}{PointPointer}\PYG{o}{\PYGZlt{}}\PYG{k+kt}{double}\PYG{o}{\PYGZgt{}} \PYG{n}{point}\PYG{p}{)}\PYG{p}{\PYGZob{}}
    \PYG{o}{*}\PYG{n}{point} \PYG{o}{=} \PYG{n}{mltk}\PYG{o}{:}\PYG{o}{:}\PYG{n}{normalize}\PYG{p}{(}\PYG{o}{*}\PYG{n}{point}\PYG{p}{,} \PYG{l+m+mi}{2}\PYG{p}{)}\PYG{p}{;}
\PYG{p}{\PYGZcb{}}\PYG{p}{;}

\PYG{n}{data}\PYG{p}{.}\PYG{n}{apply}\PYG{p}{(}\PYG{n}{normalization}\PYG{p}{)}\PYG{p}{;}
\end{sphinxVerbatim}

\sphinxAtStartPar
Initializing a data object with 10 uniform distributed random points:

\begin{sphinxVerbatim}[commandchars=\\\{\}]
\PYG{n}{mltk}\PYG{o}{:}\PYG{o}{:}\PYG{n}{Data}\PYG{o}{\PYGZlt{}}\PYG{o}{\PYGZgt{}} \PYG{n}{data}\PYG{p}{;}
\PYG{k}{for}\PYG{p}{(}\PYG{k+kt}{int} \PYG{n}{i} \PYG{o}{=} \PYG{l+m+mi}{0}\PYG{p}{;} \PYG{n}{i} \PYG{o}{\PYGZlt{}} \PYG{l+m+mi}{10}\PYG{p}{;} \PYG{n}{i}\PYG{o}{+}\PYG{o}{+}\PYG{p}{)}\PYG{p}{\PYGZob{}}
    \PYG{k}{auto} \PYG{n}{p}  \PYG{o}{=} \PYG{n}{mltk}\PYG{o}{:}\PYG{o}{:}\PYG{n}{random\PYGZus{}init}\PYG{p}{(}\PYG{l+m+mi}{3}\PYG{p}{,} \PYG{l+m+mi}{42}\PYG{p}{)}\PYG{p}{;}

    \PYG{n}{data}\PYG{p}{.}\PYG{n}{insertPoint}\PYG{p}{(}\PYG{n}{p}\PYG{p}{)}\PYG{p}{;}
\PYG{p}{\PYGZcb{}}
\end{sphinxVerbatim}

\sphinxAtStartPar
Below are the methods of insertion/removal:
\begin{itemize}
\item {} 
\sphinxAtStartPar
insertPoint \sphinxhyphen{} insert a point to the dataset;

\item {} 
\sphinxAtStartPar
removePoint \sphinxhyphen{} remove a point with the given unique id;

\item {} 
\sphinxAtStartPar
removeFeature \sphinxhyphen{} remove a dimension with the given id (1..dim) from the dataset.

\end{itemize}

\sphinxAtStartPar
You can see the concepts presented here in practice on the implementation of \sphinxhref{https://github.com/mateus558/UFJF-Machine-Learning-Toolkit/blob/main/subprojects/ufjfmltk-core/src/Datasets.cpp}{algorithms for artificial datasets generation}.


\chapter{Classification}
\label{\detokenize{getting_started/classification:classification}}\label{\detokenize{getting_started/classification::doc}}
\sphinxAtStartPar
Often we are given the task, from ourselves or from others, to label things according to a set of already existing classes:
\begin{itemize}
\item {} 
\sphinxAtStartPar
Is the object in the image a vehicle or a cat?

\item {} 
\sphinxAtStartPar
Is this animal a dog or a cat?

\end{itemize}

\sphinxAtStartPar
\sphinxstyleemphasis{Classification} is the problem of giving the right label to a record given as input. The task is different from regression because
here we have discrete labels instead of continuous values \sphinxcite{getting_started/classification:skiena2017}. In this chapter we’ll give a brief introduction on binary
and multi\sphinxhyphen{}class classification tasks and show how to tackle these problems using \sphinxstylestrong{UFJF\sphinxhyphen{}MLTK}.

\sphinxAtStartPar
Add \sphinxcode{\sphinxupquote{\#include \textless{}ufjfmltk/Classification.hpp\textgreater{}}} to include the classification module.


\section{Binary classification}
\label{\detokenize{getting_started/classification:binary-classification}}
\begin{figure}[htbp]
\centering
\capstart

\noindent\sphinxincludegraphics[width=400\sphinxpxdimen]{{binclass}.png}
\caption{Example of a binary classification problem with a linear discriminant.}\label{\detokenize{getting_started/classification:id14}}\end{figure}

\sphinxAtStartPar
Let \(Z = (x_{i}, y_{i})\) be a set of samples of size \(m\), where \(x_{i} \in R^{d}\), called input space of the problem,
\(y_{i}\) is a scalar representing the class of each vector \(x_{i}\) and for binary classification \(y_{i} \in \{+1,-1\}\),
for \(i = \{1, \dots, m\}\). A linear classifier, in a linearly separable input space, is represented by a hyperplane with the following equation \sphinxcite{getting_started/classification:villela2011}:
\begin{equation*}
\begin{split}h(x) = \langle w, x \rangle + b\end{split}
\end{equation*}
\sphinxAtStartPar
The classification result can be obtained through a signal function \(\varphi\) applied to the discriminant value associated to the hyperplane equation, i.e:
\begin{equation*}
\begin{split}\varphi (h(x)) =
\begin{cases}
  +1,& \text{if } h(x) \geq 0\\
  -1,& \text{otherwise}
\end{cases}\end{split}
\end{equation*}

\subsection{The Perceptron algorithm}
\label{\detokenize{getting_started/classification:the-perceptron-algorithm}}
\sphinxAtStartPar
Considered the first learning algorithm, the Perceptron model is a pattern recognition model proposed by \sphinxcite{getting_started/classification:rosenblatt1958}. It’s structured by
a input layer connecting each input unit to a component from a \(d\)\sphinxhyphen{}dimension vector, and a output layer composed of \(m\) units.
Therefore, it’s an artificial neural network model with only one processing layer. In its simplest form, the Perceptron algorithm is a classification
algorithm involving only two classes \sphinxcite{getting_started/classification:villela2011}.

\begin{figure}[htbp]
\centering
\capstart

\noindent\sphinxincludegraphics[width=300\sphinxpxdimen]{{perceptron-topology}.png}
\caption{Perceptron model topology.}\label{\detokenize{getting_started/classification:id15}}\end{figure}

\sphinxAtStartPar
The algorithm developed by Rosenblatt can be utilized to determine the \(w\) vector in a limited number of iterations, where the number of
iterations is related to the number of updates of the weights vector. As the weights vector \(w\) is determined by successive corrections in order
to minimize a loss function, we can say that the separating hyperplane is constructed in a iterative way characterizing an \sphinxstyleemphasis{online} learning process \sphinxcite{getting_started/classification:villela2011}.
\sphinxSetupCaptionForVerbatim{Primal Perceptron example}
\def\sphinxLiteralBlockLabel{\label{\detokenize{getting_started/classification:primal-perc}}}
\fvset{hllines={, 9, 11,}}%
\begin{sphinxVerbatim}[commandchars=\\\{\}]
\PYG{c+cp}{\PYGZsh{}}\PYG{c+cp}{include} \PYG{c+cpf}{\PYGZlt{}ufjfmltk/ufjfmltk.hpp\PYGZgt{}}

\PYG{k}{namespace} \PYG{n}{vis} \PYG{o}{=} \PYG{n}{mltk}\PYG{o}{:}\PYG{o}{:}\PYG{n}{visualize}\PYG{p}{;}
\PYG{k}{namespace} \PYG{n}{classifier} \PYG{o}{=} \PYG{n}{mltk}\PYG{o}{:}\PYG{o}{:}\PYG{n}{classifier}\PYG{p}{;}

\PYG{k+kt}{int} \PYG{n+nf}{main}\PYG{p}{(}\PYG{p}{)} \PYG{p}{\PYGZob{}}
    \PYG{n}{mltk}\PYG{o}{:}\PYG{o}{:}\PYG{n}{Data}\PYG{o}{\PYGZlt{}}\PYG{k+kt}{double}\PYG{o}{\PYGZgt{}} \PYG{n}{data}\PYG{p}{(}\PYG{l+s}{\PYGZdq{}}\PYG{l+s}{iris.data}\PYG{l+s}{\PYGZdq{}}\PYG{p}{)}\PYG{p}{;}
    \PYG{n}{vis}\PYG{o}{:}\PYG{o}{:}\PYG{n}{Visualization}\PYG{o}{\PYGZlt{}}\PYG{o}{\PYGZgt{}} \PYG{n}{vis}\PYG{p}{(}\PYG{n}{data}\PYG{p}{)}\PYG{p}{;}
    \PYG{n}{classifier}\PYG{o}{:}\PYG{o}{:}\PYG{n}{PerceptronPrimal}\PYG{o}{\PYGZlt{}}\PYG{k+kt}{double}\PYG{o}{\PYGZgt{}} \PYG{n}{perceptron}\PYG{p}{(}\PYG{n}{data}\PYG{p}{)}\PYG{p}{;}

    \PYG{n}{perceptron}\PYG{p}{.}\PYG{n}{train}\PYG{p}{(}\PYG{p}{)}\PYG{p}{;}

    \PYG{n}{vis}\PYG{p}{.}\PYG{n}{plot2DwithHyperplane}\PYG{p}{(}\PYG{l+m+mi}{1}\PYG{p}{,} \PYG{l+m+mi}{2}\PYG{p}{,} \PYG{n}{perceptron}\PYG{p}{.}\PYG{n}{getSolution}\PYG{p}{(}\PYG{p}{)}\PYG{p}{,} \PYG{n+nb}{true}\PYG{p}{)}\PYG{p}{;}

    \PYG{k}{return} \PYG{l+m+mi}{0}\PYG{p}{;}
\PYG{p}{\PYGZcb{}}
\end{sphinxVerbatim}
\sphinxresetverbatimhllines

\sphinxAtStartPar
On \hyperref[\detokenize{getting_started/classification:primal-perc}]{Listing \ref{\detokenize{getting_started/classification:primal-perc}}} we can see a simple usage of the \sphinxstylestrong{UFJF\sphinxhyphen{}MLTK} perceptron implementation in it’s primal form. In this example we first
load the binary \sphinxcode{\sphinxupquote{iris.data}} dataset where two of the three original classes were merged into one in order to generate a binary problem, after that we instantiate
the \sphinxcode{\sphinxupquote{PerceptronPrimal}} wrapper with the same data type as the dataset and the default parameters. With the object from the algorithm wrapper we call the
method \sphinxcode{\sphinxupquote{train}} to learn a model from the data and, finally, the decision boundary is ploted with features 1 and 2 from the dataset and passing the perceptron solution. \hyperref[\detokenize{getting_started/classification:primal-perc-hyp}]{Fig.\@ \ref{\detokenize{getting_started/classification:primal-perc-hyp}}}
shows the hyperplane generated by the model.

\begin{figure}[htbp]
\centering
\capstart

\noindent\sphinxincludegraphics[width=450\sphinxpxdimen]{{ima-iris-2dsol}.png}
\caption{Solution generated from the model trained by the Perceptron classifier.}\label{\detokenize{getting_started/classification:primal-perc-hyp}}\end{figure}


\section{Kernel methods}
\label{\detokenize{getting_started/classification:kernel-methods}}
\sphinxAtStartPar
Often in real datasets is not possible to do a linear separation of the data. In these cases is necessary
to utilize more complex functions for labels separation. One way to define a non\sphinxhyphen{}linear separator is through
a mapping function from input space \(X\) to a higher dimensional space where the separation is possible \sphinxcite{getting_started/classification:mehryar2018}.

\sphinxAtStartPar
In models based on a mapping from the fixed non\sphinxhyphen{}linear features space \(\Phi(x)\), the kernel function is defined
as following \sphinxcite{getting_started/classification:bishop2007}:
\begin{equation}\label{equation:getting_started/classification:kernel-func}
\begin{split}k(x, x^{'}) = \Phi(x)^{T}\Phi(x^{'})\end{split}
\end{equation}
\sphinxAtStartPar
\hyperref[\detokenize{getting_started/classification:spirals-data}]{Fig.\@ \ref{\detokenize{getting_started/classification:spirals-data}}} shows an example of a dataset that isn’t linearly separable. It’s composed of two spirals and as we can see, there isn’t a way to
draw a line that separates the samples belonging to each spiral. In the {\hyperref[\detokenize{getting_started/classification:the-perceptron-dual-algorithm}]{\emph{Dual Perceptron}}} section we’ll see how to solve this problem.

\begin{figure}[htbp]
\centering
\capstart

\noindent\sphinxincludegraphics[width=450\sphinxpxdimen]{{spirals}.png}
\caption{Spirals artificial dataset.}\label{\detokenize{getting_started/classification:spirals-data}}\end{figure}

\sphinxAtStartPar
The simplest kernel considering the mapping on Eq. \eqref{equation:getting_started/classification:kernel-func} is the linear kernel where
\(\Phi(x) = x\) and \(k(x, x^{'}) = x^{T}x\). The kernel concept formulated as a inner product in the
input space allows the generalization of many known algorithms. The main idea is that if an algorithm is formulated
in such a way that the input vector \(x\) is presented in a scalar product form, the inner product can be replaced
by another kernel product. This kind of extension is known as \sphinxstylestrong{kernel trick} or kernel substitution \sphinxcite{getting_started/classification:bishop2007}.


\subsection{The Perceptron dual algorithm}
\label{\detokenize{getting_started/classification:the-perceptron-dual-algorithm}}
\sphinxAtStartPar
The derivation and implementation of the dual form of the Perceptron algorithm will be shown in Section ??, since it’s a more complex topic. For now,
we’ll use \sphinxstylestrong{UFJF\sphinxhyphen{}MLTK} implementation to solve the spirals dataset problem presented earlier.
\sphinxSetupCaptionForVerbatim{Dual perceptron training on spirals artificial dataset.}
\def\sphinxLiteralBlockLabel{\label{\detokenize{getting_started/classification:dualperc-spirals}}}
\fvset{hllines={, 9, 11, 12,}}%
\begin{sphinxVerbatim}[commandchars=\\\{\}]
  \PYG{c+cp}{\PYGZsh{}}\PYG{c+cp}{include} \PYG{c+cpf}{\PYGZlt{}ufjfmltk/ufjfmltk.hpp\PYGZgt{}}

  \PYG{k}{namespace} \PYG{n}{vis} \PYG{o}{=} \PYG{n}{mltk}\PYG{o}{:}\PYG{o}{:}\PYG{n}{visualize}\PYG{p}{;}
  \PYG{k}{namespace} \PYG{n}{classifier} \PYG{o}{=} \PYG{n}{mltk}\PYG{o}{:}\PYG{o}{:}\PYG{n}{classifier}\PYG{p}{;}

  \PYG{k+kt}{int} \PYG{n+nf}{main}\PYG{p}{(}\PYG{p}{)} \PYG{p}{\PYGZob{}}
      \PYG{k}{auto} \PYG{n}{data} \PYG{o}{=} \PYG{n}{mltk}\PYG{o}{:}\PYG{o}{:}\PYG{n}{datasets}\PYG{o}{:}\PYG{o}{:}\PYG{n}{make\PYGZus{}spirals}\PYG{p}{(}\PYG{l+m+mi}{500}\PYG{p}{)}\PYG{p}{;}
      \PYG{n}{vis}\PYG{o}{:}\PYG{o}{:}\PYG{n}{Visualization}\PYG{o}{\PYGZlt{}}\PYG{o}{\PYGZgt{}} \PYG{n}{vis}\PYG{p}{(}\PYG{n}{data}\PYG{p}{)}\PYG{p}{;}
      \PYG{n}{classifier}\PYG{o}{:}\PYG{o}{:}\PYG{n}{PerceptronDual}\PYG{o}{\PYGZlt{}}\PYG{k+kt}{double}\PYG{o}{\PYGZgt{}} \PYG{n}{perceptron}\PYG{p}{(}\PYG{n}{data}\PYG{p}{,} \PYG{n}{mltk}\PYG{o}{:}\PYG{o}{:}\PYG{n}{KernelType}\PYG{o}{:}\PYG{o}{:}\PYG{n}{GAUSSIAN}\PYG{p}{,} \PYG{l+m+mi}{1}\PYG{p}{)}\PYG{p}{;}

      \PYG{n}{perceptron}\PYG{p}{.}\PYG{n}{setMaxTime}\PYG{p}{(}\PYG{l+m+mi}{500}\PYG{p}{)}\PYG{p}{;}
      \PYG{n}{perceptron}\PYG{p}{.}\PYG{n}{train}\PYG{p}{(}\PYG{p}{)}\PYG{p}{;}

      \PYG{n}{vis}\PYG{p}{.}\PYG{n}{plotDecisionSurface2D}\PYG{p}{(}\PYG{n}{perceptron}\PYG{p}{,} \PYG{l+m+mi}{0}\PYG{p}{,} \PYG{l+m+mi}{1}\PYG{p}{,} \PYG{n+nb}{true}\PYG{p}{,} \PYG{l+m+mi}{100}\PYG{p}{)}\PYG{p}{;}
      \PYG{k}{return} \PYG{l+m+mi}{0}\PYG{p}{;}
  \PYG{p}{\PYGZcb{}}
\end{sphinxVerbatim}
\sphinxresetverbatimhllines

\sphinxAtStartPar
\hyperref[\detokenize{getting_started/classification:dualperc-spirals}]{Listing \ref{\detokenize{getting_started/classification:dualperc-spirals}}} example generates a spirals
dataset with 500 samples using the \sphinxcode{\sphinxupquote{make\_spirals}} function from \sphinxcode{\sphinxupquote{mltk::datasets::}} namespace, initialize the visualization object and instantiate the \sphinxcode{\sphinxupquote{PerceptronDual}}
wrapper with a gaussian kernel with standard deviation of 1.0 as a kernel parameter. To guarantee the algorithm convergence, the maximum training time of the algorithm
is set as 500ms, after that, the model is trained and its decision boundary is ploted as in \hyperref[\detokenize{getting_started/classification:spirals-dualperc-dec}]{Fig.\@ \ref{\detokenize{getting_started/classification:spirals-dualperc-dec}}}.

\begin{figure}[htbp]
\centering
\capstart

\noindent\sphinxincludegraphics[width=450\sphinxpxdimen]{{contour-spirals-percdual}.png}
\caption{Decision contour surface from Perceptron dual for spirals dataset.}\label{\detokenize{getting_started/classification:spirals-dualperc-dec}}\end{figure}


\section{Multi\sphinxhyphen{}class classification}
\label{\detokenize{getting_started/classification:multi-class-classification}}
\sphinxAtStartPar
Until now we’ve been discussing algorithms for classification problems were we have only two labels, but often we face problems where we need
to choose a class between tens, hundreds or even thousands of labels, like when we need to assign a label to an object in an image. In this chapter, we’ll
be analysing the problem of multi\sphinxhyphen{}class classification learning.

\sphinxAtStartPar
Let \(\mathcal{X}\) be the input space and \(\mathcal{Y}\) the output space, and let \(\mathcal{D}\) be an unknown distribution over \(\mathcal{X}\) according
to which input points are drawn. We’ll be distinguishing between the \sphinxstyleemphasis{mono\sphinxhyphen{}label} (binary classification) and \sphinxstyleemphasis{multi\sphinxhyphen{}label} cases, where we define \(\mathcal{Y}\) as a set
of discrete values as \(\mathcal{Y} = \{1, \dots, k\}\) and \(\mathcal{Y} = \{+1, -1\}^{k}\) for the \sphinxstyleemphasis{mono\sphinxhyphen{}label} and \sphinxstyleemphasis{multi\sphinxhyphen{}label} cases, respectively. In the \sphinxstyleemphasis{mono\sphinxhyphen{}label} case,
each sample will be assigned to only one class, while in the \sphinxstyleemphasis{multi\sphinxhyphen{}label} there can be several. The latter can be illustrated as the positive value being the component of a vector
representing the classes where the example is associated \sphinxcite{getting_started/classification:mehryar2018}.

\sphinxAtStartPar
On both cases, the learner receives labeled samples \(\mathcal{S} = ((x_1, y_1), \dots, (x_m, y_m)) \in (\mathcal{X}, \mathcal{Y})^{m}\) with \(x_1, \dots, x_m\) drawn according
to \(\mathcal{D}\), and \(y_i = f(x_i)\) for all \(i \in [1, \dots, m]\), where \(f:\mathcal{X} \rightarrow \mathcal{Y}\) is the target labeling function. The multi\sphinxhyphen{}class classification problem consists
of using labeled data \(\mathcal{S}\) to find a hypothesis \(h \in H\), where \(H\) is a hypothesis set containing functions mapping \(\mathcal{X}\) to \(\mathcal{Y}\). The multi\sphinxhyphen{}class classification problem consists on finding the hypothesis \(h \in H\) using the labeled data \(\mathcal{S}\), such that
it has smallest generalization error \(R(h)\) with respect to the target \(f\), where Eq. \eqref{equation:getting_started/classification:mono} refers to the \sphinxstyleemphasis{mono\sphinxhyphen{}label} case and Eq. \eqref{equation:getting_started/classification:multi} to the \sphinxstyleemphasis{multi\sphinxhyphen{}label} case \sphinxcite{getting_started/classification:mehryar2018}.
\begin{equation}\label{equation:getting_started/classification:mono}
\begin{split}R(h) = \mathop{\mathbb{E}}_{x \sim \mathcal{D}} [1_{h(x) \neq f(x)}]\end{split}
\end{equation}\begin{equation}\label{equation:getting_started/classification:multi}
\begin{split}R(h) = \mathop{\mathbb{E}}_{x \sim \mathcal{D}} [\sum_{l=1}^{k} 1_{[h(x)]_l \neq [f(x)]_l}]\end{split}
\end{equation}
\sphinxAtStartPar
In the following sections we’ll be discussing two algorithms for adapting models for binary classification to the multi\sphinxhyphen{}class case, namely One\sphinxhyphen{}vs\sphinxhyphen{}All and One\sphinxhyphen{}vs\sphinxhyphen{}One. For that,
the blobs artificial dataset generated with 50 examples for each of 3 labels. The plot for the dataset data can be seen on \hyperref[\detokenize{getting_started/classification:blobs-3class}]{Fig.\@ \ref{\detokenize{getting_started/classification:blobs-3class}}}.

\begin{figure}[htbp]
\centering
\capstart

\noindent\sphinxincludegraphics[width=450\sphinxpxdimen]{{blobs}.png}
\caption{Blobs artificial dataset.}\label{\detokenize{getting_started/classification:blobs-3class}}\end{figure}


\subsection{The One\sphinxhyphen{}vs\sphinxhyphen{}All algorithm}
\label{\detokenize{getting_started/classification:the-one-vs-all-algorithm}}
\sphinxAtStartPar
This method consists in learning \(k\) binary classifiers \(h_l:\mathcal{X} \rightarrow {-1, +1}\), \(l \in \mathcal{Y}\), each one of them
designed to discriminate one class from all the others. Each \(h_l\), for any \(l \in \mathcal{Y}\), is constructed by training a binary classifier after
relabeling points in class \(l\) with 1 and all the others as \sphinxhyphen{}1 on the full sample \(\mathcal{S}\). The multi\sphinxhyphen{}class hypothesis \(h:\mathcal{X} \rightarrow \mathcal{Y}\) defined by the
One\sphinxhyphen{}vs\sphinxhyphen{}All (OVA) technique is given by \sphinxcite{getting_started/classification:mehryar2018}:
\begin{equation*}
\begin{split}\forall x \in \mathcal{X},\; h(x) = \mathop{arg\,max}_{l\in\mathcal{Y}}f_l(x)\end{split}
\end{equation*}
\sphinxAtStartPar
\hyperref[\detokenize{getting_started/classification:ova-example}]{Listing \ref{\detokenize{getting_started/classification:ova-example}}} shows how to use the \sphinxstylestrong{UFJF\sphinxhyphen{}MLTK} primal perceptron implementation with the OVA technique to tackle the blobs dataset classification problem.
As can be seen, the only thing needed to do is to instantiate the \sphinxcode{\sphinxupquote{OneVsAll}} wrapper and pass the training data and the algorithm wrapper to be used. Something to be noted, is that
the base algorithm parameters must be passed on its initialization or before calling the OVA \sphinxcode{\sphinxupquote{train}} method.
\sphinxSetupCaptionForVerbatim{OVA example with the primal perceptron model.}
\def\sphinxLiteralBlockLabel{\label{\detokenize{getting_started/classification:ova-example}}}
\fvset{hllines={, 9, 10, 12,}}%
\begin{sphinxVerbatim}[commandchars=\\\{\}]
  \PYG{c+cp}{\PYGZsh{}}\PYG{c+cp}{include} \PYG{c+cpf}{\PYGZlt{}ufjfmltk/ufjfmltk.hpp\PYGZgt{}}

  \PYG{k}{namespace} \PYG{n}{vis} \PYG{o}{=} \PYG{n}{mltk}\PYG{o}{:}\PYG{o}{:}\PYG{n}{visualize}\PYG{p}{;}
  \PYG{k}{namespace} \PYG{n}{classifier} \PYG{o}{=} \PYG{n}{mltk}\PYG{o}{:}\PYG{o}{:}\PYG{n}{classifier}\PYG{p}{;}

  \PYG{k+kt}{int} \PYG{n+nf}{main}\PYG{p}{(}\PYG{p}{)} \PYG{p}{\PYGZob{}}
      \PYG{k}{auto} \PYG{n}{data} \PYG{o}{=} \PYG{n}{mltk}\PYG{o}{:}\PYG{o}{:}\PYG{n}{datasets}\PYG{o}{:}\PYG{o}{:}\PYG{n}{make\PYGZus{}blobs}\PYG{p}{(}\PYG{l+m+mi}{50}\PYG{p}{,} \PYG{l+m+mi}{3}\PYG{p}{,} \PYG{l+m+mi}{2}\PYG{p}{,} \PYG{l+m+mf}{1.5}\PYG{p}{,} \PYG{o}{\PYGZhy{}}\PYG{l+m+mi}{20}\PYG{p}{,} \PYG{l+m+mi}{20}\PYG{p}{,} \PYG{n+nb}{true}\PYG{p}{,} \PYG{n+nb}{true}\PYG{p}{,} \PYG{l+m+mi}{10}\PYG{p}{)}\PYG{p}{.}\PYG{n}{dataset}\PYG{p}{;}
      \PYG{n}{vis}\PYG{o}{:}\PYG{o}{:}\PYG{n}{Visualization}\PYG{o}{\PYGZlt{}}\PYG{o}{\PYGZgt{}} \PYG{n}{vis}\PYG{p}{(}\PYG{n}{data}\PYG{p}{)}\PYG{p}{;}
      \PYG{n}{classifier}\PYG{o}{:}\PYG{o}{:}\PYG{n}{PerceptronPrimal}\PYG{o}{\PYGZlt{}}\PYG{k+kt}{double}\PYG{o}{\PYGZgt{}} \PYG{n}{perceptron}\PYG{p}{;}
      \PYG{n}{classifier}\PYG{o}{:}\PYG{o}{:}\PYG{n}{OneVsAll}\PYG{o}{\PYGZlt{}}\PYG{k+kt}{double}\PYG{o}{\PYGZgt{}} \PYG{n}{ova}\PYG{p}{(}\PYG{n}{data}\PYG{p}{,} \PYG{n}{perceptron}\PYG{p}{)}\PYG{p}{;}

      \PYG{n}{ova}\PYG{p}{.}\PYG{n}{train}\PYG{p}{(}\PYG{p}{)}\PYG{p}{;}

      \PYG{n}{vis}\PYG{p}{.}\PYG{n}{plotDecisionSurface2D}\PYG{p}{(}\PYG{n}{ova}\PYG{p}{,} \PYG{l+m+mi}{0}\PYG{p}{,} \PYG{l+m+mi}{1}\PYG{p}{,} \PYG{n+nb}{true}\PYG{p}{,} \PYG{l+m+mi}{100}\PYG{p}{,} \PYG{n+nb}{true}\PYG{p}{)}\PYG{p}{;}
      \PYG{k}{return} \PYG{l+m+mi}{0}\PYG{p}{;}
  \PYG{p}{\PYGZcb{}}
\end{sphinxVerbatim}
\sphinxresetverbatimhllines

\sphinxAtStartPar
\hyperref[\detokenize{getting_started/classification:blobs-contour-ova-perc}]{Fig.\@ \ref{\detokenize{getting_started/classification:blobs-contour-ova-perc}}} shows the decision boundary generated after training, it’s possible to note that
each region drawn accomodates points with the same class, indicating that the technique was effective on learning
a aproximation of the data distribution. For non linearly separated data, the only changes is that we need
to use an algorithm capable of learning a non\sphinxhyphen{}linear function like the dual perceptron from \sphinxcode{\sphinxupquote{PerceptronDual}} wrapper.

\begin{figure}[htbp]
\centering
\capstart

\noindent\sphinxincludegraphics[width=450\sphinxpxdimen]{{contour-blobs-ova}.png}
\caption{Decision contour surface from OVA with perceptron for blobs dataset.}\label{\detokenize{getting_started/classification:blobs-contour-ova-perc}}\end{figure}


\subsection{The One\sphinxhyphen{}vs\sphinxhyphen{}One algorithm}
\label{\detokenize{getting_started/classification:the-one-vs-one-algorithm}}
\sphinxAtStartPar
The One\sphinxhyphen{}vs\sphinxhyphen{}One (OVO) technique consists in learning a binary classifier \(h_{ll^{'}}:\mathcal{X}\rightarrow {-1, +1}\) for each pair of distinct classes \((l, l^{'}) \in \mathcal{Y}\), \(l \neq l^{'}\),
discriminating \(l\) and \(l^{'}\). \(h_{ll^{'}}\) is obtained by training a binary classifier on the sub\sphinxhyphen{}sample containing exactly the points labeled as \(l\) and \(l^{'}\),
with the value +1 returned for \(l^{'}\) and \sphinxhyphen{}1 for \(l\). For that, it’s needed to train \(\binom{k}{2} = \frac{k(k-1)}{2}\) classifiers, which are combined to define a multi\sphinxhyphen{}class classification hypothesis \(h\)
via majority vote \sphinxcite{getting_started/classification:mehryar2018}:
\begin{equation*}
\begin{split}\forall x \in \mathcal{X},\; h(x) = \mathop{arg\,max}_{l^{'} \in \mathcal{Y}}| \{l:h_{ll^{'}}(x) = 1\} |\end{split}
\end{equation*}\sphinxSetupCaptionForVerbatim{OVO example with the primal perceptron model.}
\def\sphinxLiteralBlockLabel{\label{\detokenize{getting_started/classification:ovo-example}}}
\fvset{hllines={, 9, 10, 12,}}%
\begin{sphinxVerbatim}[commandchars=\\\{\}]
  \PYG{c+cp}{\PYGZsh{}}\PYG{c+cp}{include} \PYG{c+cpf}{\PYGZlt{}ufjfmltk/ufjfmltk.hpp\PYGZgt{}}

  \PYG{k}{namespace} \PYG{n}{vis} \PYG{o}{=} \PYG{n}{mltk}\PYG{o}{:}\PYG{o}{:}\PYG{n}{visualize}\PYG{p}{;}
  \PYG{k}{namespace} \PYG{n}{classifier} \PYG{o}{=} \PYG{n}{mltk}\PYG{o}{:}\PYG{o}{:}\PYG{n}{classifier}\PYG{p}{;}

  \PYG{k+kt}{int} \PYG{n+nf}{main}\PYG{p}{(}\PYG{p}{)} \PYG{p}{\PYGZob{}}
      \PYG{k}{auto} \PYG{n}{data} \PYG{o}{=} \PYG{n}{mltk}\PYG{o}{:}\PYG{o}{:}\PYG{n}{datasets}\PYG{o}{:}\PYG{o}{:}\PYG{n}{make\PYGZus{}blobs}\PYG{p}{(}\PYG{l+m+mi}{50}\PYG{p}{,} \PYG{l+m+mi}{3}\PYG{p}{,} \PYG{l+m+mi}{2}\PYG{p}{,} \PYG{l+m+mf}{1.5}\PYG{p}{,} \PYG{o}{\PYGZhy{}}\PYG{l+m+mi}{20}\PYG{p}{,} \PYG{l+m+mi}{20}\PYG{p}{,} \PYG{n+nb}{true}\PYG{p}{,} \PYG{n+nb}{true}\PYG{p}{,} \PYG{l+m+mi}{10}\PYG{p}{)}\PYG{p}{.}\PYG{n}{dataset}\PYG{p}{;}
      \PYG{n}{vis}\PYG{o}{:}\PYG{o}{:}\PYG{n}{Visualization}\PYG{o}{\PYGZlt{}}\PYG{o}{\PYGZgt{}} \PYG{n}{vis}\PYG{p}{(}\PYG{n}{data}\PYG{p}{)}\PYG{p}{;}
      \PYG{n}{classifier}\PYG{o}{:}\PYG{o}{:}\PYG{n}{PerceptronPrimal}\PYG{o}{\PYGZlt{}}\PYG{k+kt}{double}\PYG{o}{\PYGZgt{}} \PYG{n}{perceptron}\PYG{p}{;}
      \PYG{n}{classifier}\PYG{o}{:}\PYG{o}{:}\PYG{n}{OneVsOne}\PYG{o}{\PYGZlt{}}\PYG{k+kt}{double}\PYG{o}{\PYGZgt{}} \PYG{n}{ovo}\PYG{p}{(}\PYG{n}{data}\PYG{p}{,} \PYG{n}{perceptron}\PYG{p}{)}\PYG{p}{;}

      \PYG{n}{ovo}\PYG{p}{.}\PYG{n}{train}\PYG{p}{(}\PYG{p}{)}\PYG{p}{;}

      \PYG{n}{vis}\PYG{p}{.}\PYG{n}{plotDecisionSurface2D}\PYG{p}{(}\PYG{n}{ovo}\PYG{p}{,} \PYG{l+m+mi}{0}\PYG{p}{,} \PYG{l+m+mi}{1}\PYG{p}{,} \PYG{n+nb}{true}\PYG{p}{,} \PYG{l+m+mi}{100}\PYG{p}{,} \PYG{n+nb}{true}\PYG{p}{)}\PYG{p}{;}

      \PYG{k}{return} \PYG{l+m+mi}{0}\PYG{p}{;}
  \PYG{p}{\PYGZcb{}}
\end{sphinxVerbatim}
\sphinxresetverbatimhllines

\sphinxAtStartPar
\hyperref[\detokenize{getting_started/classification:ovo-example}]{Listing \ref{\detokenize{getting_started/classification:ovo-example}}} is analogous to \hyperref[\detokenize{getting_started/classification:ova-example}]{Listing \ref{\detokenize{getting_started/classification:ova-example}}} except that it’s using the \sphinxcode{\sphinxupquote{OneVsOne}} wrapper instead of the OVA one.
As expected, it could also learn the data distribution, this can be seen by the decision boundary shown at \hyperref[\detokenize{getting_started/classification:blobs-contour-ovo-perc}]{Fig.\@ \ref{\detokenize{getting_started/classification:blobs-contour-ovo-perc}}}.

\begin{figure}[htbp]
\centering
\capstart

\noindent\sphinxincludegraphics[width=450\sphinxpxdimen]{{contour-blobs-ovo}.png}
\caption{Decision contour surface from OVO with perceptron for blobs dataset.}\label{\detokenize{getting_started/classification:blobs-contour-ovo-perc}}\end{figure}


\section{Model evaluation and selection}
\label{\detokenize{getting_started/classification:model-evaluation-and-selection}}
\sphinxAtStartPar
So far, you may have been able to build a classifier, but only that is not enough. Supose you’ve trained a model to predict the purchasing behavior of future clients using data from
previous sales. For that, you need to estimate how accurately your model can be on unseen data, i.e, how accurately your model can predict the behavior of future customers. You may have built
several classifiers and need to compare how well they can be between each other \sphinxcite{getting_started/classification:han2011}. This section address metrics that can be used to compare those methods and how reliable this comparison can be.


\subsection{Holdout method and random subsampling}
\label{\detokenize{getting_started/classification:holdout-method-and-random-subsampling}}

\subsection{Cross\sphinxhyphen{}validation}
\label{\detokenize{getting_started/classification:cross-validation}}

\chapter{Extending the framework}
\label{\detokenize{contribute/extending:extending-the-framework}}\label{\detokenize{contribute/extending::doc}}

\section{Implementing classifiers}
\label{\detokenize{contribute/extending:implementing-classifiers}}
\sphinxAtStartPar
TODO


\subsection{Perceptron primal algorithm}
\label{\detokenize{contribute/extending:perceptron-primal-algorithm}}
\sphinxAtStartPar
TODO


\subsection{Perceptron dual algorithm}
\label{\detokenize{contribute/extending:perceptron-dual-algorithm}}

\chapter{Contributor Covenant Code of Conduct}
\label{\detokenize{contribute/codeconduct:contributor-covenant-code-of-conduct}}\label{\detokenize{contribute/codeconduct::doc}}

\section{Our Pledge}
\label{\detokenize{contribute/codeconduct:our-pledge}}
\sphinxAtStartPar
In the interest of fostering an open and welcoming environment, we as contributors and maintainers pledge to making participation in our project and our community a harassment\sphinxhyphen{}free experience for everyone, regardless of age, body size, disability, ethnicity, gender identity and expression, level of experience, nationality, personal appearance, race, religion, or sexual identity and orientation.


\section{Our Standards}
\label{\detokenize{contribute/codeconduct:our-standards}}
\sphinxAtStartPar
Examples of behavior that contributes to creating a positive environment include:
\begin{itemize}
\item {} 
\sphinxAtStartPar
Using welcoming and inclusive language

\item {} 
\sphinxAtStartPar
Being respectful of differing viewpoints and experiences

\item {} 
\sphinxAtStartPar
Gracefully accepting constructive criticism

\item {} 
\sphinxAtStartPar
Focusing on what is best for the community

\item {} 
\sphinxAtStartPar
Showing empathy towards other community members

\end{itemize}

\sphinxAtStartPar
Examples of unacceptable behavior by participants include:
\begin{itemize}
\item {} 
\sphinxAtStartPar
The use of sexualized language or imagery and unwelcome sexual attention or advances

\item {} 
\sphinxAtStartPar
Trolling, insulting/derogatory comments, and personal or political attacks

\item {} 
\sphinxAtStartPar
Public or private harassment

\item {} 
\sphinxAtStartPar
Publishing others’ private information, such as a physical or electronic address, without explicit permission

\item {} 
\sphinxAtStartPar
Other conduct which could reasonably be considered inappropriate in a professional setting

\end{itemize}


\section{Our Responsibilities}
\label{\detokenize{contribute/codeconduct:our-responsibilities}}
\sphinxAtStartPar
Project maintainers are responsible for clarifying the standards of acceptable behavior and are expected to take appropriate and fair corrective action in response to any instances of unacceptable behavior.

\sphinxAtStartPar
Project maintainers have the right and responsibility to remove, edit, or reject comments, commits, code, wiki edits, issues, and other contributions that are not aligned to this Code of Conduct, or to ban temporarily or permanently any contributor for other behaviors that they deem inappropriate, threatening, offensive, or harmful.


\section{Scope}
\label{\detokenize{contribute/codeconduct:scope}}
\sphinxAtStartPar
This Code of Conduct applies both within project spaces and in public spaces when an individual is representing the project or its community. Examples of representing a project or community include using an official project e\sphinxhyphen{}mail address, posting via an official social media account, or acting as an appointed representative at an online or offline event. Representation of a project may be further defined and clarified by project maintainers.


\section{Enforcement}
\label{\detokenize{contribute/codeconduct:enforcement}}
\sphinxAtStartPar
Instances of abusive, harassing, or otherwise unacceptable behavior may be reported by contacting the project team at \sphinxhref{mailto:mateus.marim@ice.ufjf.br}{mateus.marim@ice.ufjf.br}. The project team will review and investigate all complaints, and will respond in a way that it deems appropriate to the circumstances. The project team is obligated to maintain confidentiality with regard to the reporter of an incident. Further details of specific enforcement policies may be posted separately.

\sphinxAtStartPar
Project maintainers who do not follow or enforce the Code of Conduct in good faith may face temporary or permanent repercussions as determined by other members of the project’s leadership.


\section{Attribution}
\label{\detokenize{contribute/codeconduct:attribution}}
\sphinxAtStartPar
This Code of Conduct is adapted from the \sphinxstylestrong{Contributor Covenant} \sphinxhref{http://contributor-covenant.org}{homepage}, version 1.4, available at \sphinxhref{http://contributor-covenant.org/version/1/4/}{version}


\chapter{GNU GENERAL PUBLIC LICENSE}
\label{\detokenize{license:gnu-general-public-license}}\label{\detokenize{license::doc}}
\sphinxAtStartPar
\sphinxstyleemphasis{Version 3, 29 June 2007}

\sphinxAtStartPar
Copyright (C) 2007 Free Software Foundation, Inc. \textless{}\sphinxurl{https://fsf.org/}\textgreater{}
Everyone is permitted to copy and distribute verbatim copies of this license document, but changing it is not allowed.


\section{Preamble}
\label{\detokenize{license:preamble}}
\sphinxAtStartPar
The GNU General Public License is a free, copyleft license for software and other kinds of works.

\sphinxAtStartPar
The licenses for most software and other practical works are designed
to take away your freedom to share and change the works.  By contrast,
the GNU General Public License is intended to guarantee your freedom to
share and change all versions of a program\textendash{}to make sure it remains free
software for all its users.  We, the Free Software Foundation, use the
GNU General Public License for most of our software; it applies also to
any other work released this way by its authors.  You can apply it to
your programs, too.

\sphinxAtStartPar
When we speak of free software, we are referring to freedom, not price.  Our General Public Licenses are designed to make sure that you have the freedom to distribute copies of free software (and charge for
them if you wish), that you receive source code or can get it if you
want it, that you can change the software or use pieces of it in new
free programs, and that you know you can do these things.

\sphinxAtStartPar
To protect your rights, we need to prevent others from denying you
these rights or asking you to surrender the rights.  Therefore, you have
certain responsibilities if you distribute copies of the software, or if
you modify it: responsibilities to respect the freedom of others.

\sphinxAtStartPar
For example, if you distribute copies of such a program, whether
gratis or for a fee, you must pass on to the recipients the same
freedoms that you received.  You must make sure that they, too, receive
or can get the source code.  And you must show them these terms so they
know their rights.

\sphinxAtStartPar
Developers that use the GNU GPL protect your rights with two steps:
(1) assert copyright on the software, and (2) offer you this License
giving you legal permission to copy, distribute and/or modify it.

\sphinxAtStartPar
For the developers’ and authors’ protection, the GPL clearly explains
that there is no warranty for this free software.  For both users’ and
authors’ sake, the GPL requires that modified versions be marked as
changed, so that their problems will not be attributed erroneously to
authors of previous versions.

\sphinxAtStartPar
Some devices are designed to deny users access to install or run
modified versions of the software inside them, although the manufacturer
can do so.  This is fundamentally incompatible with the aim of
protecting users’ freedom to change the software.  The systematic
pattern of such abuse occurs in the area of products for individuals to
use, which is precisely where it is most unacceptable.  Therefore, we
have designed this version of the GPL to prohibit the practice for those
products.  If such problems arise substantially in other domains, we
stand ready to extend this provision to those domains in future versions
of the GPL, as needed to protect the freedom of users.

\sphinxAtStartPar
Finally, every program is threatened constantly by software patents.
States should not allow patents to restrict development and use of
software on general\sphinxhyphen{}purpose computers, but in those that do, we wish to
avoid the special danger that patents applied to a free program could
make it effectively proprietary.  To prevent this, the GPL assures that
patents cannot be used to render the program non\sphinxhyphen{}free.

\sphinxAtStartPar
The precise terms and conditions for copying, distribution and
modification follow.


\section{TERMS AND CONDITIONS}
\label{\detokenize{license:terms-and-conditions}}

\subsection{0. Definitions.}
\label{\detokenize{license:definitions}}
\sphinxAtStartPar
“This License” refers to version 3 of the GNU General Public License.

\sphinxAtStartPar
“Copyright” also means copyright\sphinxhyphen{}like laws that apply to other kinds of
works, such as semiconductor masks.

\sphinxAtStartPar
“The Program” refers to any copyrightable work licensed under this
License.  Each licensee is addressed as “you”.  “Licensees” and
“recipients” may be individuals or organizations.

\sphinxAtStartPar
To “modify” a work means to copy from or adapt all or part of the work
in a fashion requiring copyright permission, other than the making of an
exact copy.  The resulting work is called a “modified version” of the
earlier work or a work “based on” the earlier work.

\sphinxAtStartPar
A “covered work” means either the unmodified Program or a work based
on the Program.

\sphinxAtStartPar
To “propagate” a work means to do anything with it that, without
permission, would make you directly or secondarily liable for
infringement under applicable copyright law, except executing it on a
computer or modifying a private copy.  Propagation includes copying,
distribution (with or without modification), making available to the
public, and in some countries other activities as well.

\sphinxAtStartPar
To “convey” a work means any kind of propagation that enables other
parties to make or receive copies.  Mere interaction with a user through
a computer network, with no transfer of a copy, is not conveying.

\sphinxAtStartPar
An interactive user interface displays “Appropriate Legal Notices”
to the extent that it includes a convenient and prominently visible
feature that (1) displays an appropriate copyright notice, and (2)
tells the user that there is no warranty for the work (except to the
extent that warranties are provided), that licensees may convey the
work under this License, and how to view a copy of this License.  If
the interface presents a list of user commands or options, such as a
menu, a prominent item in the list meets this criterion.


\subsection{1. Source Code.}
\label{\detokenize{license:source-code}}
\sphinxAtStartPar
The “source code” for a work means the preferred form of the work
for making modifications to it.  “Object code” means any non\sphinxhyphen{}source
form of a work.

\sphinxAtStartPar
A “Standard Interface” means an interface that either is an official
standard defined by a recognized standards body, or, in the case of
interfaces specified for a particular programming language, one that
is widely used among developers working in that language.

\sphinxAtStartPar
The “System Libraries” of an executable work include anything, other
than the work as a whole, that (a) is included in the normal form of
packaging a Major Component, but which is not part of that Major
Component, and (b) serves only to enable use of the work with that
Major Component, or to implement a Standard Interface for which an
implementation is available to the public in source code form.  A
“Major Component”, in this context, means a major essential component
(kernel, window system, and so on) of the specific operating system
(if any) on which the executable work runs, or a compiler used to
produce the work, or an object code interpreter used to run it.

\sphinxAtStartPar
The “Corresponding Source” for a work in object code form means all
the source code needed to generate, install, and (for an executable
work) run the object code and to modify the work, including scripts to
control those activities.  However, it does not include the work’s
System Libraries, or general\sphinxhyphen{}purpose tools or generally available free
programs which are used unmodified in performing those activities but
which are not part of the work.  For example, Corresponding Source
includes interface definition files associated with source files for
the work, and the source code for shared libraries and dynamically
linked subprograms that the work is specifically designed to require,
such as by intimate data communication or control flow between those
subprograms and other parts of the work.

\sphinxAtStartPar
The Corresponding Source need not include anything that users
can regenerate automatically from other parts of the Corresponding
Source.

\sphinxAtStartPar
The Corresponding Source for a work in source code form is that
same work.


\subsection{2. Basic Permissions.}
\label{\detokenize{license:basic-permissions}}
\sphinxAtStartPar
All rights granted under this License are granted for the term of
copyright on the Program, and are irrevocable provided the stated
conditions are met.  This License explicitly affirms your unlimited
permission to run the unmodified Program.  The output from running a
covered work is covered by this License only if the output, given its
content, constitutes a covered work.  This License acknowledges your
rights of fair use or other equivalent, as provided by copyright law.

\sphinxAtStartPar
You may make, run and propagate covered works that you do not
convey, without conditions so long as your license otherwise remains
in force.  You may convey covered works to others for the sole purpose
of having them make modifications exclusively for you, or provide you
with facilities for running those works, provided that you comply with
the terms of this License in conveying all material for which you do
not control copyright.  Those thus making or running the covered works
for you must do so exclusively on your behalf, under your direction
and control, on terms that prohibit them from making any copies of
your copyrighted material outside their relationship with you.

\sphinxAtStartPar
Conveying under any other circumstances is permitted solely under
the conditions stated below.  Sublicensing is not allowed; section 10
makes it unnecessary.


\subsection{3. Protecting Users’ Legal Rights From Anti\sphinxhyphen{}Circumvention Law.}
\label{\detokenize{license:protecting-users-legal-rights-from-anti-circumvention-law}}
\sphinxAtStartPar
No covered work shall be deemed part of an effective technological
measure under any applicable law fulfilling obligations under article
11 of the WIPO copyright treaty adopted on 20 December 1996, or
similar laws prohibiting or restricting circumvention of such
measures.

\sphinxAtStartPar
When you convey a covered work, you waive any legal power to forbid
circumvention of technological measures to the extent such circumvention
is effected by exercising rights under this License with respect to
the covered work, and you disclaim any intention to limit operation or
modification of the work as a means of enforcing, against the work’s
users, your or third parties’ legal rights to forbid circumvention of
technological measures.


\subsection{4. Conveying Verbatim Copies.}
\label{\detokenize{license:conveying-verbatim-copies}}
\sphinxAtStartPar
You may convey verbatim copies of the Program’s source code as you
receive it, in any medium, provided that you conspicuously and
appropriately publish on each copy an appropriate copyright notice;
keep intact all notices stating that this License and any
non\sphinxhyphen{}permissive terms added in accord with section 7 apply to the code;
keep intact all notices of the absence of any warranty; and give all
recipients a copy of this License along with the Program.

\sphinxAtStartPar
You may charge any price or no price for each copy that you convey,
and you may offer support or warranty protection for a fee.


\subsection{5. Conveying Modified Source Versions.}
\label{\detokenize{license:conveying-modified-source-versions}}
\sphinxAtStartPar
You may convey a work based on the Program, or the modifications to
produce it from the Program, in the form of source code under the
terms of section 4, provided that you also meet all of these conditions:
\begin{quote}

\sphinxAtStartPar
a) The work must carry prominent notices stating that you modified
it, and giving a relevant date.

\sphinxAtStartPar
b) The work must carry prominent notices stating that it is
released under this License and any conditions added under section
7.  This requirement modifies the requirement in section 4 to
“keep intact all notices”.

\sphinxAtStartPar
c) You must license the entire work, as a whole, under this
License to anyone who comes into possession of a copy.  This
License will therefore apply, along with any applicable section 7
additional terms, to the whole of the work, and all its parts,
regardless of how they are packaged.  This License gives no
permission to license the work in any other way, but it does not
invalidate such permission if you have separately received it.

\sphinxAtStartPar
d) If the work has interactive user interfaces, each must display
Appropriate Legal Notices; however, if the Program has interactive
interfaces that do not display Appropriate Legal Notices, your
work need not make them do so.
\end{quote}

\sphinxAtStartPar
A compilation of a covered work with other separate and independent
works, which are not by their nature extensions of the covered work,
and which are not combined with it such as to form a larger program,
in or on a volume of a storage or distribution medium, is called an
“aggregate” if the compilation and its resulting copyright are not
used to limit the access or legal rights of the compilation’s users
beyond what the individual works permit.  Inclusion of a covered work
in an aggregate does not cause this License to apply to the other
parts of the aggregate.


\subsection{6. Conveying Non\sphinxhyphen{}Source Forms.}
\label{\detokenize{license:conveying-non-source-forms}}
\sphinxAtStartPar
You may convey a covered work in object code form under the terms
of sections 4 and 5, provided that you also convey the
machine\sphinxhyphen{}readable Corresponding Source under the terms of this License,
in one of these ways:
\begin{quote}

\sphinxAtStartPar
a) Convey the object code in, or embodied in, a physical product
(including a physical distribution medium), accompanied by the
Corresponding Source fixed on a durable physical medium
customarily used for software interchange.

\sphinxAtStartPar
b) Convey the object code in, or embodied in, a physical product
(including a physical distribution medium), accompanied by a
written offer, valid for at least three years and valid for as
long as you offer spare parts or customer support for that product
model, to give anyone who possesses the object code either (1) a
copy of the Corresponding Source for all the software in the
product that is covered by this License, on a durable physical
medium customarily used for software interchange, for a price no
more than your reasonable cost of physically performing this
conveying of source, or (2) access to copy the
Corresponding Source from a network server at no charge.

\sphinxAtStartPar
c) Convey individual copies of the object code with a copy of the
written offer to provide the Corresponding Source.  This
alternative is allowed only occasionally and noncommercially, and
only if you received the object code with such an offer, in accord
with subsection 6b.

\sphinxAtStartPar
d) Convey the object code by offering access from a designated
place (gratis or for a charge), and offer equivalent access to the
Corresponding Source in the same way through the same place at no
further charge.  You need not require recipients to copy the
Corresponding Source along with the object code.  If the place to
copy the object code is a network server, the Corresponding Source
may be on a different server (operated by you or a third party)
that supports equivalent copying facilities, provided you maintain
clear directions next to the object code saying where to find the
Corresponding Source.  Regardless of what server hosts the
Corresponding Source, you remain obligated to ensure that it is
available for as long as needed to satisfy these requirements.

\sphinxAtStartPar
e) Convey the object code using peer\sphinxhyphen{}to\sphinxhyphen{}peer transmission, provided
you inform other peers where the object code and Corresponding
Source of the work are being offered to the general public at no
charge under subsection 6d.
\end{quote}

\sphinxAtStartPar
A separable portion of the object code, whose source code is excluded
from the Corresponding Source as a System Library, need not be
included in conveying the object code work.

\sphinxAtStartPar
A “User Product” is either (1) a “consumer product”, which means any
tangible personal property which is normally used for personal, family,
or household purposes, or (2) anything designed or sold for incorporation
into a dwelling.  In determining whether a product is a consumer product,
doubtful cases shall be resolved in favor of coverage.  For a particular
product received by a particular user, “normally used” refers to a
typical or common use of that class of product, regardless of the status
of the particular user or of the way in which the particular user
actually uses, or expects or is expected to use, the product.  A product
is a consumer product regardless of whether the product has substantial
commercial, industrial or non\sphinxhyphen{}consumer uses, unless such uses represent
the only significant mode of use of the product.

\sphinxAtStartPar
“Installation Information” for a User Product means any methods,
procedures, authorization keys, or other information required to install
and execute modified versions of a covered work in that User Product from
a modified version of its Corresponding Source.  The information must
suffice to ensure that the continued functioning of the modified object
code is in no case prevented or interfered with solely because
modification has been made.

\sphinxAtStartPar
If you convey an object code work under this section in, or with, or
specifically for use in, a User Product, and the conveying occurs as
part of a transaction in which the right of possession and use of the
User Product is transferred to the recipient in perpetuity or for a
fixed term (regardless of how the transaction is characterized), the
Corresponding Source conveyed under this section must be accompanied
by the Installation Information.  But this requirement does not apply
if neither you nor any third party retains the ability to install
modified object code on the User Product (for example, the work has
been installed in ROM).

\sphinxAtStartPar
The requirement to provide Installation Information does not include a
requirement to continue to provide support service, warranty, or updates
for a work that has been modified or installed by the recipient, or for
the User Product in which it has been modified or installed.  Access to a
network may be denied when the modification itself materially and
adversely affects the operation of the network or violates the rules and
protocols for communication across the network.

\sphinxAtStartPar
Corresponding Source conveyed, and Installation Information provided,
in accord with this section must be in a format that is publicly
documented (and with an implementation available to the public in
source code form), and must require no special password or key for
unpacking, reading or copying.


\subsection{7. Additional Terms.}
\label{\detokenize{license:additional-terms}}
\sphinxAtStartPar
“Additional permissions” are terms that supplement the terms of this
License by making exceptions from one or more of its conditions.
Additional permissions that are applicable to the entire Program shall
be treated as though they were included in this License, to the extent
that they are valid under applicable law.  If additional permissions
apply only to part of the Program, that part may be used separately
under those permissions, but the entire Program remains governed by
this License without regard to the additional permissions.

\sphinxAtStartPar
When you convey a copy of a covered work, you may at your option
remove any additional permissions from that copy, or from any part of
it.  (Additional permissions may be written to require their own
removal in certain cases when you modify the work.)  You may place
additional permissions on material, added by you to a covered work,
for which you have or can give appropriate copyright permission.

\sphinxAtStartPar
Notwithstanding any other provision of this License, for material you
add to a covered work, you may (if authorized by the copyright holders of
that material) supplement the terms of this License with terms:
\begin{quote}

\sphinxAtStartPar
a) Disclaiming warranty or limiting liability differently from the
terms of sections 15 and 16 of this License; or

\sphinxAtStartPar
b) Requiring preservation of specified reasonable legal notices or
author attributions in that material or in the Appropriate Legal
Notices displayed by works containing it; or

\sphinxAtStartPar
c) Prohibiting misrepresentation of the origin of that material, or
requiring that modified versions of such material be marked in
reasonable ways as different from the original version; or

\sphinxAtStartPar
d) Limiting the use for publicity purposes of names of licensors or
authors of the material; or

\sphinxAtStartPar
e) Declining to grant rights under trademark law for use of some
trade names, trademarks, or service marks; or

\sphinxAtStartPar
f) Requiring indemnification of licensors and authors of that
material by anyone who conveys the material (or modified versions of
it) with contractual assumptions of liability to the recipient, for
any liability that these contractual assumptions directly impose on
those licensors and authors.
\end{quote}

\sphinxAtStartPar
All other non\sphinxhyphen{}permissive additional terms are considered “further
restrictions” within the meaning of section 10.  If the Program as you
received it, or any part of it, contains a notice stating that it is
governed by this License along with a term that is a further
restriction, you may remove that term.  If a license document contains
a further restriction but permits relicensing or conveying under this
License, you may add to a covered work material governed by the terms
of that license document, provided that the further restriction does
not survive such relicensing or conveying.

\sphinxAtStartPar
If you add terms to a covered work in accord with this section, you
must place, in the relevant source files, a statement of the
additional terms that apply to those files, or a notice indicating
where to find the applicable terms.

\sphinxAtStartPar
Additional terms, permissive or non\sphinxhyphen{}permissive, may be stated in the
form of a separately written license, or stated as exceptions;
the above requirements apply either way.


\subsection{8. Termination.}
\label{\detokenize{license:termination}}
\sphinxAtStartPar
You may not propagate or modify a covered work except as expressly
provided under this License.  Any attempt otherwise to propagate or
modify it is void, and will automatically terminate your rights under
this License (including any patent licenses granted under the third
paragraph of section 11).

\sphinxAtStartPar
However, if you cease all violation of this License, then your
license from a particular copyright holder is reinstated (a)
provisionally, unless and until the copyright holder explicitly and
finally terminates your license, and (b) permanently, if the copyright
holder fails to notify you of the violation by some reasonable means
prior to 60 days after the cessation.

\sphinxAtStartPar
Moreover, your license from a particular copyright holder is
reinstated permanently if the copyright holder notifies you of the
violation by some reasonable means, this is the first time you have
received notice of violation of this License (for any work) from that
copyright holder, and you cure the violation prior to 30 days after
your receipt of the notice.

\sphinxAtStartPar
Termination of your rights under this section does not terminate the
licenses of parties who have received copies or rights from you under
this License.  If your rights have been terminated and not permanently
reinstated, you do not qualify to receive new licenses for the same
material under section 10.


\subsection{9. Acceptance Not Required for Having Copies.}
\label{\detokenize{license:acceptance-not-required-for-having-copies}}
\sphinxAtStartPar
You are not required to accept this License in order to receive or
run a copy of the Program.  Ancillary propagation of a covered work
occurring solely as a consequence of using peer\sphinxhyphen{}to\sphinxhyphen{}peer transmission
to receive a copy likewise does not require acceptance.  However,
nothing other than this License grants you permission to propagate or
modify any covered work.  These actions infringe copyright if you do
not accept this License.  Therefore, by modifying or propagating a
covered work, you indicate your acceptance of this License to do so.


\subsection{10. Automatic Licensing of Downstream Recipients.}
\label{\detokenize{license:automatic-licensing-of-downstream-recipients}}
\sphinxAtStartPar
Each time you convey a covered work, the recipient automatically
receives a license from the original licensors, to run, modify and
propagate that work, subject to this License.  You are not responsible
for enforcing compliance by third parties with this License.

\sphinxAtStartPar
An “entity transaction” is a transaction transferring control of an
organization, or substantially all assets of one, or subdividing an
organization, or merging organizations.  If propagation of a covered
work results from an entity transaction, each party to that
transaction who receives a copy of the work also receives whatever
licenses to the work the party’s predecessor in interest had or could
give under the previous paragraph, plus a right to possession of the
Corresponding Source of the work from the predecessor in interest, if
the predecessor has it or can get it with reasonable efforts.

\sphinxAtStartPar
You may not impose any further restrictions on the exercise of the
rights granted or affirmed under this License.  For example, you may
not impose a license fee, royalty, or other charge for exercise of
rights granted under this License, and you may not initiate litigation
(including a cross\sphinxhyphen{}claim or counterclaim in a lawsuit) alleging that
any patent claim is infringed by making, using, selling, offering for
sale, or importing the Program or any portion of it.


\subsection{11. Patents.}
\label{\detokenize{license:patents}}
\sphinxAtStartPar
A “contributor” is a copyright holder who authorizes use under this
License of the Program or a work on which the Program is based.  The
work thus licensed is called the contributor’s “contributor version”.

\sphinxAtStartPar
A contributor’s “essential patent claims” are all patent claims
owned or controlled by the contributor, whether already acquired or
hereafter acquired, that would be infringed by some manner, permitted
by this License, of making, using, or selling its contributor version,
but do not include claims that would be infringed only as a
consequence of further modification of the contributor version.  For
purposes of this definition, “control” includes the right to grant
patent sublicenses in a manner consistent with the requirements of
this License.

\sphinxAtStartPar
Each contributor grants you a non\sphinxhyphen{}exclusive, worldwide, royalty\sphinxhyphen{}free
patent license under the contributor’s essential patent claims, to
make, use, sell, offer for sale, import and otherwise run, modify and
propagate the contents of its contributor version.

\sphinxAtStartPar
In the following three paragraphs, a “patent license” is any express
agreement or commitment, however denominated, not to enforce a patent
(such as an express permission to practice a patent or covenant not to
sue for patent infringement).  To “grant” such a patent license to a
party means to make such an agreement or commitment not to enforce a
patent against the party.

\sphinxAtStartPar
If you convey a covered work, knowingly relying on a patent license,
and the Corresponding Source of the work is not available for anyone
to copy, free of charge and under the terms of this License, through a
publicly available network server or other readily accessible means,
then you must either (1) cause the Corresponding Source to be so
available, or (2) arrange to deprive yourself of the benefit of the
patent license for this particular work, or (3) arrange, in a manner
consistent with the requirements of this License, to extend the patent
license to downstream recipients.  “Knowingly relying” means you have
actual knowledge that, but for the patent license, your conveying the
covered work in a country, or your recipient’s use of the covered work
in a country, would infringe one or more identifiable patents in that
country that you have reason to believe are valid.

\sphinxAtStartPar
If, pursuant to or in connection with a single transaction or
arrangement, you convey, or propagate by procuring conveyance of, a
covered work, and grant a patent license to some of the parties
receiving the covered work authorizing them to use, propagate, modify
or convey a specific copy of the covered work, then the patent license
you grant is automatically extended to all recipients of the covered
work and works based on it.

\sphinxAtStartPar
A patent license is “discriminatory” if it does not include within
the scope of its coverage, prohibits the exercise of, or is
conditioned on the non\sphinxhyphen{}exercise of one or more of the rights that are
specifically granted under this License.  You may not convey a covered
work if you are a party to an arrangement with a third party that is
in the business of distributing software, under which you make payment
to the third party based on the extent of your activity of conveying
the work, and under which the third party grants, to any of the
parties who would receive the covered work from you, a discriminatory
patent license (a) in connection with copies of the covered work
conveyed by you (or copies made from those copies), or (b) primarily
for and in connection with specific products or compilations that
contain the covered work, unless you entered into that arrangement,
or that patent license was granted, prior to 28 March 2007.

\sphinxAtStartPar
Nothing in this License shall be construed as excluding or limiting
any implied license or other defenses to infringement that may
otherwise be available to you under applicable patent law.


\subsection{12. No Surrender of Others’ Freedom.}
\label{\detokenize{license:no-surrender-of-others-freedom}}
\sphinxAtStartPar
If conditions are imposed on you (whether by court order, agreement or
otherwise) that contradict the conditions of this License, they do not
excuse you from the conditions of this License.  If you cannot convey a
covered work so as to satisfy simultaneously your obligations under this
License and any other pertinent obligations, then as a consequence you may
not convey it at all.  For example, if you agree to terms that obligate you
to collect a royalty for further conveying from those to whom you convey
the Program, the only way you could satisfy both those terms and this
License would be to refrain entirely from conveying the Program.


\subsection{13. Use with the GNU Affero General Public License.}
\label{\detokenize{license:use-with-the-gnu-affero-general-public-license}}
\sphinxAtStartPar
Not with standing any other provision of this License, you have
permission to link or combine any covered work with a work licensed
under version 3 of the GNU Affero General Public License into a single
combined work, and to convey the resulting work.  The terms of this
License will continue to apply to the part which is the covered work,
but the special requirements of the GNU Affero General Public License,
section 13, concerning interaction through a network will apply to the
combination as such.


\subsection{14. Revised Versions of this License.}
\label{\detokenize{license:revised-versions-of-this-license}}
\sphinxAtStartPar
The Free Software Foundation may publish revised and/or new versions of
the GNU General Public License from time to time.  Such new versions will
be similar in spirit to the present version, but may differ in detail to
address new problems or concerns.

\sphinxAtStartPar
Each version is given a distinguishing version number.  If the
Program specifies that a certain numbered version of the GNU General
Public License “or any later version” applies to it, you have the
option of following the terms and conditions either of that numbered
version or of any later version published by the Free Software
Foundation.  If the Program does not specify a version number of the
GNU General Public License, you may choose any version ever published
by the Free Software Foundation.

\sphinxAtStartPar
If the Program specifies that a proxy can decide which future
versions of the GNU General Public License can be used, that proxy’s
public statement of acceptance of a version permanently authorizes you
to choose that version for the Program.

\sphinxAtStartPar
Later license versions may give you additional or different
permissions.  However, no additional obligations are imposed on any
author or copyright holder as a result of your choosing to follow a
later version.


\subsection{15. Disclaimer of Warranty.}
\label{\detokenize{license:disclaimer-of-warranty}}
\sphinxAtStartPar
THERE IS NO WARRANTY FOR THE PROGRAM, TO THE EXTENT PERMITTED BY
APPLICABLE LAW.  EXCEPT WHEN OTHERWISE STATED IN WRITING THE COPYRIGHT
HOLDERS AND/OR OTHER PARTIES PROVIDE THE PROGRAM “AS IS” WITHOUT WARRANTY
OF ANY KIND, EITHER EXPRESSED OR IMPLIED, INCLUDING, BUT NOT LIMITED TO,
THE IMPLIED WARRANTIES OF MERCHANTABILITY AND FITNESS FOR A PARTICULAR
PURPOSE.  THE ENTIRE RISK AS TO THE QUALITY AND PERFORMANCE OF THE PROGRAM
IS WITH YOU.  SHOULD THE PROGRAM PROVE DEFECTIVE, YOU ASSUME THE COST OF
ALL NECESSARY SERVICING, REPAIR OR CORRECTION.


\subsection{16. Limitation of Liability.}
\label{\detokenize{license:limitation-of-liability}}
\sphinxAtStartPar
IN NO EVENT UNLESS REQUIRED BY APPLICABLE LAW OR AGREED TO IN WRITING
WILL ANY COPYRIGHT HOLDER, OR ANY OTHER PARTY WHO MODIFIES AND/OR CONVEYS
THE PROGRAM AS PERMITTED ABOVE, BE LIABLE TO YOU FOR DAMAGES, INCLUDING ANY
GENERAL, SPECIAL, INCIDENTAL OR CONSEQUENTIAL DAMAGES ARISING OUT OF THE
USE OR INABILITY TO USE THE PROGRAM (INCLUDING BUT NOT LIMITED TO LOSS OF
DATA OR DATA BEING RENDERED INACCURATE OR LOSSES SUSTAINED BY YOU OR THIRD
PARTIES OR A FAILURE OF THE PROGRAM TO OPERATE WITH ANY OTHER PROGRAMS),
EVEN IF SUCH HOLDER OR OTHER PARTY HAS BEEN ADVISED OF THE POSSIBILITY OF
SUCH DAMAGES.


\subsection{17. Interpretation of Sections 15 and 16.}
\label{\detokenize{license:interpretation-of-sections-15-and-16}}
\sphinxAtStartPar
If the disclaimer of warranty and limitation of liability provided
above cannot be given local legal effect according to their terms,
reviewing courts shall apply local law that most closely approximates
an absolute waiver of all civil liability in connection with the
Program, unless a warranty or assumption of liability accompanies a
copy of the Program in return for a fee.
\begin{quote}

\sphinxAtStartPar
\sphinxstylestrong{END OF TERMS AND CONDITIONS}
\end{quote}


\section{How to Apply These Terms to Your New Programs}
\label{\detokenize{license:how-to-apply-these-terms-to-your-new-programs}}
\sphinxAtStartPar
If you develop a new program, and you want it to be of the greatest
possible use to the public, the best way to achieve this is to make it
free software which everyone can redistribute and change under these terms.

\sphinxAtStartPar
To do so, attach the following notices to the program.  It is safest
to attach them to the start of each source file to most effectively
state the exclusion of warranty; and each file should have at least
the “copyright” line and a pointer to where the full notice is found.
\begin{quote}

\sphinxAtStartPar
\textless{}one line to give the program’s name and a brief idea of what it does.\textgreater{}
Copyright (C) \textless{}year\textgreater{}  \textless{}name of author\textgreater{}

\sphinxAtStartPar
This program is free software: you can redistribute it and/or modify
it under the terms of the GNU General Public License as published by
the Free Software Foundation, either version 3 of the License, or
(at your option) any later version.

\sphinxAtStartPar
This program is distributed in the hope that it will be useful,
but WITHOUT ANY WARRANTY; without even the implied warranty of
MERCHANTABILITY or FITNESS FOR A PARTICULAR PURPOSE.  See the
GNU General Public License for more details.

\sphinxAtStartPar
You should have received a copy of the GNU General Public License
along with this program.  If not, see \textless{}\sphinxurl{https://www.gnu.org/licenses/}\textgreater{}.
\end{quote}

\sphinxAtStartPar
Also add information on how to contact you by electronic and paper mail.

\sphinxAtStartPar
If the program does terminal interaction, make it output a short
notice like this when it starts in an interactive mode:
\begin{quote}

\sphinxAtStartPar
\textless{}program\textgreater{}  Copyright (C) \textless{}year\textgreater{}  \textless{}name of author\textgreater{}
This program comes with ABSOLUTELY NO WARRANTY; for details type {\color{red}\bfseries{}\textasciigrave{}}show w’.
This is free software, and you are welcome to redistribute it
under certain conditions; type {\color{red}\bfseries{}\textasciigrave{}}show c’ for details.
\end{quote}

\sphinxAtStartPar
The hypothetical commands {\color{red}\bfseries{}\textasciigrave{}}show w’ and {\color{red}\bfseries{}\textasciigrave{}}show c’ should show the appropriate
parts of the General Public License.  Of course, your program’s commands
might be different; for a GUI interface, you would use an “about box”.

\sphinxAtStartPar
You should also get your employer (if you work as a programmer) or school,
if any, to sign a “copyright disclaimer” for the program, if necessary.
For more information on this, and how to apply and follow the GNU GPL, see
\textless{}\sphinxurl{https://www.gnu.org/licenses/}\textgreater{}.

\sphinxAtStartPar
The GNU General Public License does not permit incorporating your program
into proprietary programs.  If your program is a subroutine library, you
may consider it more useful to permit linking proprietary applications with
the library.  If this is what you want to do, use the GNU Lesser General
Public License instead of this License.  But first, please read
\textless{}\sphinxurl{https://www.gnu.org/licenses/why-not-lgpl.html}\textgreater{}.

\begin{sphinxthebibliography}{ROSENBLA}
\bibitem[SKIENA2017]{getting_started/classification:skiena2017}
\sphinxAtStartPar
Skiena, Steven S. The data science design manual. Springer, 2017.
\bibitem[VILLELA2011]{getting_started/classification:villela2011}
\sphinxAtStartPar
Villela, Saulo Moraes, et al. “Seleção de Características utilizando Busca Ordenada e um Classificador de Larga Margem.” (2011).
\bibitem[ROSENBLATT1958]{getting_started/classification:rosenblatt1958}
\sphinxAtStartPar
Rosenblatt, Frank. “The perceptron: a probabilistic model for information storage and organization in the brain.” Psychological review 65.6 (1958): 386.
\bibitem[MEHRYAR2018]{getting_started/classification:mehryar2018}
\sphinxAtStartPar
Mohri, Mehryar, Afshin Rostamizadeh, and Ameet Talwalkar. Foundations of machine learning. MIT press, 2018.
\bibitem[BISHOP2007]{getting_started/classification:bishop2007}
\sphinxAtStartPar
Bishop, Christopher M. “Pattern recognition and machine learning (information science and statistics).” (2007).
\bibitem[HAN2011]{getting_started/classification:han2011}
\sphinxAtStartPar
Han, Jiawei, Jian Pei, and Micheline Kamber. Data mining: concepts and techniques. Elsevier, 2011.
\end{sphinxthebibliography}



\renewcommand{\indexname}{Index}
\printindex
\end{document}